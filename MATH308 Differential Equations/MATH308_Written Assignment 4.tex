\documentclass[10pt]{article}

\usepackage{geometry}
\geometry{margin = .5in,top =0.12\paperheight, headheight=\paperheight}
\usepackage{array}
\usepackage{amsmath}
\usepackage{amsfonts}
\usepackage{amsthm}
\usepackage[export]{adjustbox}
\usepackage{fancyhdr}
\usepackage{lastpage}
\pagestyle{fancy}
\fancyhf{}
\rhead{Written Assignment, Page \thepage}
\lhead{MATH308}
\chead{\includegraphics[width = 0.15\textwidth]{MCLogo-Bck.png}}
\usepackage{mathrsfs}

%\renewcommand{\footrulewidth}{0.4pt}

\usepackage{enumitem}
\usepackage{pifont}
\usepackage{graphicx}
\graphicspath{{../img}}

\newtheorem*{theorem}{Theorem}
\newtheorem{exercise}{Exercise}


\newcommand{\R}{\mathbb R}
\newcommand{\e}{{\rm e}}
\newcommand{\inpr}[1]{\left\langle#1\right\rangle}
\newcommand{\norm}[1]{\lVert #1 \rVert}
\newcommand{\abs}[1]{\lvert #1 \rvert}
\newcommand{\vv}{\mathbf v}
\newcommand{\uv}{\mathbf u}

\DeclareMathOperator{\xd}{d\!}

\title{}
\date{}

\begin{document}
Consider the $n$th order linear homogeneous equation with constant coefficients:
\[
	L(y) =y^{(n)} +  \sum_{i = 0}^{n - 1}a_iy^{(i)} = 0.
\]
Suppose $\phi$ is a solution to this equation. Define
\[
\norm{\phi(x)} = \left(\sum_{i = 0}^{n - 1} {\left(\phi^{(i)}(x)\right)^2}\right)^{1/2}.
\]
We begin with finding an estimate of $\norm{\phi(x)}$ in some explicit terms. Let $u(x) = \norm{\phi(x)}^2 = \sum_{i = 0}^{n - 1} {\left(\phi^{(i)}(x)\right)^2}$, then

\begin{align*}
u'(x) &=\hspace{10em} \\[2em]
\abs{u'(x)}& \leq\\
\end{align*}
Since $\phi$ is assumed to be a solution of $L(y) = 0$, we have

\begin{align*}
	\phi^{(n)}(x) &= \sum_{i = 0}^{n - 1} -a_i\phi^{(i)}(x)\\
	\abs{\phi^{(n)}(x)} &\leq  \sum_{i = 0}^{n - 1} \abs{a_i}\abs{\phi^{(i)}(x)}
\end{align*}
Consequently, substituting in $u'(x)$, we have
\vspace{\stretch{1}}\\
We now apply the elementary inequality $2\abs{b}\abs{c} \leq \abs{b}^2 + \abs{c}^2$
to obtain
\vspace{\stretch{3}}\\
Now, denote $k =1 + \sum_{i = 0}^{n - 1}\abs{a_i}$, then 
\[
\abs{u'} \leq 2ku,
\]
or equivalently, 
\begin{equation}
\label{uIneq}
-2ku \leq {u'} \leq 2ku.
\end{equation}
\clearpage
\noindent
{\bf Problems.}
\begin{enumerate}
\item
Based on inequality (\ref{uIneq}), prove that
\begin{equation}
\label{phiIneq}
	\norm{\phi(x_0)}e^{-k\abs{x - x_0}} \leq \norm{\phi(x)}\leq \norm{\phi(x_0)}e^{k\abs{x - x_0}}.
\end{equation}
\vspace{\stretch{3}}
\item Let $\{c_i\}_{i=0}^{n-1}$ be any $n$ constants, and let $x_0$ be any real number. We want to prove that there exists at most one solution $\phi$  of $L(y) = 0$ satisfying $\{y^{(i)}(x_0) = c_i\}_{i=0}^{n-1}$. We break the proof into two steps:
\begin{enumerate}
\item
  Suppose both $\phi$ and $\psi$ were two solutions of $L(y) = 0$ satisfying $\{y^{(i)}(x_0) = c_i\}_{i=0}^{n-1}$. Prove that $\chi = \phi - \psi$ satisfies $L(y) = 0$ and $\{\chi^{(i)}(x_0) = 0\}_{i=0}^{n-1}$.
  \vspace{\stretch{1}}
\item Use inequality (\ref{phiIneq}) to prove that $\norm{\chi} = 0$, which implies $\chi(x) = 0$ for all $x$ and hence $\phi = \psi$.
\vspace{\stretch{1}}
\end{enumerate}

\end{enumerate}
\end{document}