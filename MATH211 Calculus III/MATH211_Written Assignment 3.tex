\documentclass[10pt]{article}

\usepackage{geometry}
\geometry{margin = 1in,top =0.12\paperheight, headheight=\paperheight}
\usepackage[export]{adjustbox}
\usepackage{array}
\usepackage{amsmath}
\usepackage{amsfonts}
\usepackage{fancyhdr}
\usepackage{lastpage}
\usepackage{xcolor}
\usepackage{comment}
\pagestyle{fancy}
\fancyhf{}
\rhead{Written Assignment, Page \thepage}
\lhead{MATH211}
\chead{\includegraphics[width = 0.15\textwidth]{MCLogo-Bck.png}}


%\renewcommand{\footrulewidth}{0.4pt}

\usepackage{enumitem}
\usepackage{pifont}
\usepackage{graphicx}
\graphicspath{{../img}}

\newtheorem{theorem}{Theorem}
\newtheorem{exercise}{Exercise}


\newcommand{\R}{\mathbb R}
\newcommand{\e}{{\rm e}}
\newcommand{\inpr}[1]{\left\langle#1\right\rangle}
\newcommand{\norm}[1]{\lVert #1 \rVert}
\newcommand{\abs}[1]{\lvert #1 \rvert}
\newcommand{\vv}{\mathbf v}
\newcommand{\uv}{\mathbf u}

\DeclareMathOperator{\xd}{d\!}
\DeclareMathOperator{\proj}{proj}

\title{}
\date{}

\begin{document}
\noindent
{\bf Problem.}
A {\bf couple} is defined as two parallel {forces} that have the same magnitude but opposite directions. The {torque} produced by a couple is called a {\bf couple moment}. An important property of couple moments is that the couple moment is the same with respect to any point. This problem is to prove this property. That is, in the language of mathematics:

Let $\mathbf F_1$ and $\mathbf F_2$ be two vectors in space such that $\mathbf F_2 = -\mathbf F_1$. Let $P$ and $Q$ be two fixed points. Supposed $A$ and $B$ are two arbitrary points, prove that

\[
\overrightarrow{AP}\times\mathbf F_1+\overrightarrow{AQ}\times\mathbf F_2 = \overrightarrow{BP}\times\mathbf F_1+\overrightarrow{BQ}\times\mathbf F_2.
\]
\begin{figure}[h]
\includegraphics[width=0.4\textwidth, right]{couplemoment.png}
\end{figure}

\end{document}