\documentclass[10pt]{article}

\usepackage{geometry}
\geometry{margin = 1in,top =0.12\paperheight, headheight=\paperheight}
\usepackage[export]{adjustbox}
\usepackage{array}
\usepackage{amsmath}
\usepackage{amsfonts}
\usepackage{fancyhdr}
\usepackage{lastpage}
\usepackage{xcolor}
\usepackage{comment}
%\renewcommand{\footrulewidth}{0.4pt}

\usepackage{enumitem}
\usepackage{pifont}
\usepackage{graphicx}
\graphicspath{{../img/}}

\newtheorem{theorem}{Theorem}
\newtheorem{exercise}{Exercise}


\newcommand{\R}{\mathbb R}
\newcommand{\e}{{\rm e}}
\newcommand{\inpr}[1]{\left\langle#1\right\rangle}
\newcommand{\norm}[1]{\lVert #1 \rVert}
\newcommand{\abs}[1]{\lvert #1 \rvert}
\newcommand{\vv}{\mathbf v}
\newcommand{\uv}{\mathbf u}

\DeclareMathOperator{\xd}{d\!}
\DeclareMathOperator{\proj}{proj}
\pagestyle{fancy}
\fancyhf{}
\rhead{Tutorial, Page \thepage}
\lhead{MATH211}
\chead{\includegraphics[width = 0.15\textwidth]{MCLogo-Bck.png}}

\begin{document}

\section*{Mathematica Tutorial: Solve, D, ReplaceAll, Integrate}
Mathematica is a powerful computational software that provides a wide range of functions and commands for symbolic and numerical computations. In this tutorial, we will explore four essential commands: \texttt{Solve}, \texttt{D}, \texttt{ReplaceAll}, and \texttt{Integrate}.
\subsection*{Solve}
The \texttt{Solve} command is used to find the solutions to algebraic equations or systems of equations. Given an equation or a set of equations, \texttt{Solve} returns the values of the variables that satisfy the equation(s).
Example:
\begin{align*}
\text{Solve}[x^2 - 4 == 0, x]
\end{align*}
This will return the solutions $x = \pm 2$.
\subsection*{D}
The \texttt{D} command is used to perform differentiation. It takes an expression and a variable (or a list of variables) as input and returns the derivative of the expression with respect to the specified variable(s).
Example:
\begin{align*}
\text{D}[x^3 + 2x^2 - x, x]
\end{align*}
This will return the derivative $3x^2 + 4x - 1$.
\subsection*{ReplaceAll}
The \texttt{ReplaceAll} command is used to replace parts of an expression with other expressions. It allows you to substitute one or more subexpressions within an expression with new subexpressions.
Example:
\begin{align*}
(x + y)^2 , \text{/.} , x \to a, y \to b
\end{align*}
This will replace $x$ with $a$ and $y$ with $b$, resulting in $(a + b)^2$.
\subsection*{Integrate}
The \texttt{Integrate} command is used to perform integration, both indefinite and definite. It takes an expression and a variable (or a list of variables) as input and returns the antiderivative or the definite integral of the expression with respect to the specified variable(s).
Example:
\begin{align*}
\text{Integrate}[x^2 + 2x, x]
\end{align*}
This will return the indefinite integral $\frac{x^3}{3} + x^2 + C$, where $C$ is the constant of integration.

\end{document}