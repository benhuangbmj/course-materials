\documentclass[10pt]{beamer}
\usetheme{PaloAlto}
\usecolortheme{seahorse}
\setbeamertemplate{navigation symbols}{}
\setbeamertemplate{caption}[numbered]
%general package
%\usepackage[utf8]{inputenc}
\usepackage[english]{babel}
\usepackage{geometry}
\usepackage{tcolorbox}
\usepackage[cmtip,all]{xy}
\newcommand{\longsquigarrow}{\xymatrix{{}\ar@{~>}[r]&{}}}
\usepackage[export]{adjustbox}
\usepackage{graphicx}
\graphicspath{{../img/}}
\usepackage{graphbox}
%math package
\usepackage{amsmath}
\usepackage{amsfonts}
%\usepackage{amssymb}
%\usepackage{amsthm}
%\usepackage{slashed}
%\usepackage{tikz-cd}
%\usepackage{extarrows}
%font package
\usepackage{mathrsfs}
%\usepackage{bm}
%\usepackage{thmtools}
%misc. package
\usepackage{enumitem}

\author[B.H.]{{\Large MATH211 Calculus III}\\\vspace{6pt}Instructor: Ben Huang}
\date{}
\title[Section 13.5]{Section 13.5 
Chain Rules for Functions of Several Variables}
\institute[MU]{\includegraphics[width = 0.382\textwidth]{MCLogo-Bck.png}}
\logo{\includegraphics[scale = 0.3]{MCLogo-Bck.png}}
%general package
\usepackage[utf8]{inputenc}
\usepackage[english]{babel}
\usepackage{geometry}
\usepackage{comment}

%math package
\usepackage{amsmath}
\usepackage{amsfonts}
\usepackage{amssymb}
\usepackage{amsthm}
\usepackage{slashed}
\usepackage{tikz-cd}
\usepackage{mathtools}

%font package
\usepackage{mathrsfs}
\usepackage{bm}

%misc. package
\usepackage{enumitem}
\usepackage{tcolorbox}
\usepackage{etoolbox}
\usepackage{hyperref}
\hypersetup{
  colorlinks=true, urlcolor=blue
}




%declared operators
\DeclareMathOperator{\id}{Id}%identity
\DeclareMathOperator{\ind}{Ind\!}%index
\DeclareMathOperator{\tr}{Tr}%trace
\DeclareMathOperator{\e}{e}%exponential
\DeclareMathOperator{\im}{Im\!}%image
\DeclareMathOperator{\vol}{vol}%volume
\DeclareMathOperator{\cll}{\C\ell}%complexified Clifford algebra
\DeclareMathOperator{\gd}{\slashed{\partial}}%geometric Dirac
\DeclareMathOperator{\D}{\mathcal{D}}%generalized Dirac
\DeclareMathOperator{\Div}{div}%divergence
\DeclareMathOperator{\ud}{\,\mathrm{d}\!}

\DeclareMathOperator{\Hom}{Hom}
\DeclareMathOperator{\xd}{\,d\!}
\DeclareMathOperator{\curl}{curl}
\DeclareMathOperator{\dive}{div}
\DeclareMathOperator{\proj}{proj}


\newcommand{\norm}[1]{\lVert#1\rVert}
\newcommand{\R}{\mathbb R}
\newcommand{\vF}{\mathbf F}
\newcommand{\vv}{\mathbf v}
\newcommand{\inpr}[1]{\left\langle#1\right\rangle}
\newcommand{\fix}{(a,b)}
\newcommand{\uv}{\mathbf u}
\newcommand{\abs}[1]{\lvert #1\rvert}
%texting in citation
\makeatletter
\let\cite\relax
\DeclareRobustCommand{\cite}{%
  \let\new@cite@pre\@gobble
  \@ifnextchar[\new@cite{\@citex[]}}
\def\new@cite[#1]{\@ifnextchar[{\new@citea{#1}}{\@citex[#1]}}
\def\new@citea#1{\def\new@cite@pre{#1}\@citex}
\def\@cite#1#2{[{\new@cite@pre\space#1\if\relax\detokenize{#2}\relax\else, #2\fi}]}
\makeatother

\begin{document}

\frame{\titlepage}

\begin{frame}
\frametitle{Chain Rules}
Let \(w = f(x,y)\), where \(f\) is a differentiable function of $x$  and $y$,  $x = g(t)$ and $y = h(t)$, where $g$ and $h$ are differentiable functions of $t$. Note that $w$ is a composite function of $t$. \pause
\begin{align*}
\onslide<3->{\frac{\xd w}{\xd t} &= \lim_{\Delta t \to 0} \frac{w(t+\Delta t) - w(t)}{\Delta t}\\}
\onslide<4->{& = \lim_{\Delta t \to 0} \frac{\alt<5>{{\color{red}f(x(t+\Delta t),y(t+\Delta t))  - f(x(t),y(t))}}{f(x(t+\Delta t),y(t+\Delta t))  - f(x(t),y(t))}}{\Delta t}}
\end{align*}
\onslide<5->{
Since f is differentiable, from Section 13.4,
\begin{align*}
&f(x(t+\Delta t),y(t+\Delta t))  - f(x(t),y(t))\\ =& \frac{\partial w}{\partial x}(x,y)\Delta x + \frac{\partial w}{\partial y}(x,y)\Delta y + \varepsilon_1\Delta x + \varepsilon_2\Delta y,
\end{align*}
}
\onslide <6-> {where $\Delta x  = x(t+\Delta t) - x(t)$, $\Delta y = y(t+\Delta t) - y(t)$, and $ \varepsilon_1,  \varepsilon_2 \xrightarrow{(\Delta x, \Delta y)\to(0,0)} 0$.}
\end{frame}

\begin{frame}
\frametitle{Chain Rules}
Consequently,
\begin{align*}
\alt<4->{{\color{red}\frac{\xd w}{\xd t}}}{\frac{\xd w}{\xd t}} =\ &  \frac{\partial w}{\partial x} \lim_{\Delta t \to 0}\frac{\Delta x }{\Delta t}+ \frac{\partial w}{\partial y} \lim_{\Delta t \to 0}\frac{\Delta x }{\Delta t}\\& +  \lim_{\Delta t \to 0}\varepsilon_1 \lim_{\Delta t \to 0}\frac{\Delta x }{\Delta t} +  \lim_{\Delta t \to 0}\varepsilon_2 \lim_{\Delta t \to 0}\frac{\Delta y }{\Delta t}\\
\onslide<2->{=\ &  \frac{\partial w}{\partial x} \frac{\xd x}{\xd t}  + \frac{\partial w}{\partial y} \frac{\xd y}{\xd t} + 0\cdot \frac{\xd x}{\xd t} + 0\cdot \frac{\xd y}{\xd t}\\}
\onslide<3->{=\ &  \alt<4->{{\color{red}\frac{\partial w}{\partial x} \frac{\xd x}{\xd t}  + \frac{\partial w}{\partial y} \frac{\xd y}{\xd t}}	}{\frac{\partial w}{\partial x} \frac{\xd x}{\xd t}  + \frac{\partial w}{\partial y} \frac{\xd y}{\xd t}}}
\end{align*}
\end{frame}

\end{document}