\documentclass[12pt]{article}

\usepackage{geometry}
\geometry{margin=4em}
\usepackage[export]{adjustbox}
\usepackage{array}
\usepackage{amsmath}
\usepackage{amsfonts}
\usepackage{fancyhdr}
\pagestyle{empty}
\usepackage{lastpage}
\usepackage{xcolor}
\usepackage{enumitem}
\usepackage{pifont}
\usepackage{graphicx}
\graphicspath{{../img}}
\usepackage{pgfplots}
\pgfplotsset{compat=1.18}
\usepackage{tabularx}


\newcommand{\R}{\mathbb R}
\newcommand{\e}{{\rm e}}
\newcommand{\pobr}[1]{\left\langle#1\right\rangle}
\newcommand{\norm}[1]{\lVert #1 \rVert}
\newcommand{\abs}[1]{\lvert #1 \rvert}

\DeclareMathOperator{\xd}{d\!}
\DeclareMathOperator{\proj}{proj}

\title{One Mic Time}
\date{}
\begin{document}
\begin{center}
\Huge
\bf One Mic Time

\begin{figure}[h]
\includegraphics[width=.7\textwidth, center]{one-mic-token.jpg}
\end{figure}
\end{center}
\section*{The Policy}
This is our One Mic Token. When you see it held up, it means we’re in One Mic Time. During this time, only one person talks at a time so everyone can listen and be heard.

If any noise starts while someone is speaking, we’ll pause until the room is quiet again.

If you want to share something, raise your hand and wait for a teacher to call on you. If you need to talk with a classmate—about schoolwork or anything else—you can step into the hallway with one of the co-teachers to settle it there.

\section*{The Procedure}
When the One Mic Token goes up, stop talking and raise your hand. Once the class is quiet, One Mic Time officially begins. When the teacher lowers the token, One Mic Time is over.

\section*{Conflicts}
If any conflict happens during One Mic Time, we’ll use our Conflict Handling Policy to help everyone talk it through and make peace.
\end{document}