\documentclass[12pt]{article}
\usepackage{amsmath, amssymb}
\usepackage{geometry}
\usepackage{setspace}
\usepackage{enumitem}

\geometry{margin=1in}

\begin{document}

\begin{center}
{\LARGE \textbf{Lesson Plan: Secant Slopes and Limits with \(\boldsymbol{f(x)=x^3}\)}}\\[0.5em]
{\large Precalculus \,\,|\,\, Next Generation Math Standards Aligned}
\end{center}

\vspace{1em}

\section*{Standards Alignment (Next Generation Math Standards)}
\begin{itemize}[leftmargin=2em]
    \item \textbf{F.IF.4}: Interpret key features of graphs of functions.
    \item \textbf{F.LE.1 (conceptual)}: Explore rates of change in functions.
    \item \textbf{Limit Foundations (Precalculus Extensions)}:
    Understand secant slopes as average rate of change and investigate limits graphically.
\end{itemize}

\section*{Learning Objectives}
Students will be able to:
\begin{enumerate}[leftmargin=2em]
    \item Plot and explore \(f(x)=x^3\) in GeoGebra.
    \item Select two points and compute the slope of the secant line.
    \item Express this slope as a function of the second point's \(x\)-value.
    \item Use graphing technology to observe how secant slopes behave as points converge.
    \item Develop a conceptual understanding of a limit and instantaneous rate of change.
\end{enumerate}

\section*{Materials}
\begin{itemize}[leftmargin=2em]
    \item Computers or tablets with GeoGebra
    \item Worksheet (included below)
    \item Optional: Graphing calculator
\end{itemize}

\section*{Lesson Procedure}

\subsection*{1. Launch (5 minutes)}
Review the cubic function \(f(x)=x^3\). Ask students what slope means for a curve, leading into secant lines.

\subsection*{2. GeoGebra Exploration (15 minutes)}
Students complete Items 1--3 of the worksheet:
\begin{itemize}[leftmargin=2em]
    \item Plot the function.
    \item Choose points \(A\) and \(B\).
    \item Compute the average rate of change (slope of secant line).
\end{itemize}

\subsection*{3. Slope as a Function (10 minutes)}
Students complete Item 4: Write the slope as a function of the \(x\)-value of point \(B\), then input it into GeoGebra.

\subsection*{4. Limits Through Technology (10 minutes)}
Students move point \(B\) toward point \(A\) to investigate the limiting value of secant slopes.

\subsection*{5. Summary (5 minutes)}
Discuss how the secant slope approaches a limiting value and how this idea leads to the derivative.

\section*{Assessment}
Teacher observation and worksheet completion. Optional exit ticket: Describe in one sentence what happens to the slope as two points move closer.

\section*{Homework Extension (Optional)}
Repeat the process for a different function, such as \(f(x)=x^2\) or \(f(x)=\sqrt{x}\), and compare slope limits.

\newpage

%%%%%%%%%%%%%%%%%%%%%%%%%%%%%%%%%%%%%%%%%%%%%%%%%%%%%
% WORKSHEET
%%%%%%%%%%%%%%%%%%%%%%%%%%%%%%%%%%%%%%%%%%%%%%%%%%%%%

\begin{center}
{\LARGE \textbf{Worksheet: Secant Slopes and Limits}}
\end{center}

\vspace{1em}

Let \(f(x) = x^3\).

\begin{enumerate}

\item Plot the graph of \(f(x)\) in GeoGebra.

\item Pick two random points on the graph of \(f(x)\), and plot them in GeoGebra.\\[1em]
\textbf{Point A}: \rule{10em}{0.1pt}\quad
\textbf{Point B}: \rule{10em}{0.1pt}

\item Find the slope of the line through the two points you picked. 
\vspace{\stretch{1}}

\item Write the slope as a function of the \(x\)-coordinate of Point \(B\). Then enter this function in GeoGebra.
\vspace{\stretch{1}}

\item Examine the graph of this slope function. As Point \(B\) approaches Point \(A\), what does the slope of line \(AB\) approach? What is the limit of the slope as \(B\) approaches \(A\)?
\vspace{\stretch{1}}

\end{enumerate}

\end{document}
