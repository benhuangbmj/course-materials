\documentclass[12pt]{article}
\usepackage{amsmath, amssymb}
\usepackage{geometry}
\usepackage{setspace}
\usepackage{enumitem}
\usepackage{tikz, pgfplots}
\pgfplotsset{compat=1.18}

\geometry{margin=1in}

\begin{document}
\section*{Lesson Plan: Unit Fraction Exponents}

\subsection*{Objectives}
\begin{itemize}
    \item Interpret expressions of the form $a^{1/n}$.
    \item Connect unit-fraction exponents to roots using exponent rules.
    \item Match equivalent exponential and radical expressions.
\end{itemize}

\subsection*{Warm-Up (5 minutes)}
Find a solution to each equation and check it.
\begin{enumerate}
    \item $x^2 = 25$ \hfill\underline{\hspace{2cm}}
    \item $z^2 = 7$ \hfill\underline{\hspace{2cm}}
    \item $y^3 = 8$ \hfill\underline{\hspace{2cm}}
    \item $w^3 = 19$ \hfill\underline{\hspace{2cm}}
\end{enumerate}

\subsection*{Activity 1 (15--20 minutes): Meaning of $a^{1/2}$}
Clare asks what $9^{1/2}$ means. She graphs $y=9^x$ for whole-number $x$ and estimates $9^{1/2}$.

\begin{enumerate}
    \item \textbf{Clare's Work}
    \begin{enumerate}
        \item Graph $y=9^x$ using technology. Estimate $9^{1/2}$: \underline{\hspace{3cm}}
        \item Use exponent rules to compute $9^{1/2}\cdot 9^{1/2}$: \underline{\hspace{3cm}}
        \item Conclude the value of $9^{1/2}$ must be: \underline{\hspace{3cm}}
    \end{enumerate}

    \item \textbf{Diego's Work}
    \begin{enumerate}
        \item Graph $y=3^x$ and estimate $3^{1/2}$: \underline{\hspace{3cm}}
        \item Compute $(3^{1/2})^2$ using exponent rules: \underline{\hspace{3cm}}
        \item Explain Diego's comment: ``That looks like a root!'' \underline{\hspace{3cm}}
    \end{enumerate}
\end{enumerate}

\subsection*{Activity 2 (15 minutes): Exponents and Radicals}
Match each exponential expression with an equivalent radical or numerical expression.

\begin{center}
\begin{tabular}{ll}
A.\;$7^3$ & 1.\;$\frac{1}{49}$ \\
B.\;$7^2$ & 2.\;$\frac{1}{343}$ \\
C.\;$7^1$ & 3.\;$\sqrt{7}$ \\
D.\;$7^0$ & 4.\;$\frac{1}{\sqrt[3]{7}}$ \\
E.\;$7^{-1}$ & 5.\;$\sqrt[3]{7}$ \\
F.\;$7^{-2}$ & 6.\;$49$ \\
G.\;$7^{-3}$ & 7.\;$\frac{1}{\sqrt{7}}$ \\
H.\;$7^{1/2}$ & 8.\;$343$ \\
I.\;$7^{-1/2}$ & 9.\;$7$ \\
J.\;$7^{1/3}$ & 10.\;$\frac{1}{7}$ \\
K.\;$7^{-1/3}$ & 11.\;$1$
\end{tabular}
\end{center}

\subsection*{Summary (3 minutes)}
Students articulate:  
\[
a^{1/2} = \sqrt{a}, \qquad a^{1/3} = \sqrt[3]{a}, \qquad 
a^{1/n} = \sqrt[n]{a}.
\]
These definitions preserve exponent rules.

\subsection*{Cool-Down (3--5 minutes)}
\begin{enumerate}
    \item Solve $x^2 = 5$ in two ways:
    \begin{itemize}
        \item (a) using exponents: \underline{\hspace{3cm}}
        \item (b) using radicals: \underline{\hspace{3cm}}
    \end{itemize}
    \item Rewrite $\sqrt[7]{3}$ using an exponent: \underline{\hspace{3cm}}
\end{enumerate}

\end{document}
