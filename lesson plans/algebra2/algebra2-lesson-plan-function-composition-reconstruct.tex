\documentclass[12pt]{article}
\usepackage{amsmath, amssymb}
\usepackage{geometry}
\usepackage{setspace}
\usepackage{enumitem}
\usepackage{tikz, pgfplots}
\pgfplotsset{compat=1.18}

\geometry{margin=1in}

\begin{document}

%%%%%%%%%%%%%%%%%%%%%%%%%%%%%%%%%%%%%%%%%%%%%%%%%%%%%
% TITLE
%%%%%%%%%%%%%%%%%%%%%%%%%%%%%%%%%%%%%%%%%%%%%%%%%%%%%

\begin{center}
{\LARGE \textbf{Lesson Plan: Rebuilding Foundations of Function Composition}}\\[0.5em]
{\large Algebra II \,\,|\,\, Next Generation Math Standards Aligned}
\end{center}

\vspace{1em}

%%%%%%%%%%%%%%%%%%%%%%%%%%%%%%%%%%%%%%%%%%%%%%%%%%%%%
% STANDARDS
%%%%%%%%%%%%%%%%%%%%%%%%%%%%%%%%%%%%%%%%%%%%%%%%%%%%%

\section*{Standards Alignment (Next Generation Math Standards)}
\begin{itemize}[leftmargin=2em]
    \item \textbf{F.IF.1--3}: Understand functions as rules that assign inputs to outputs.
    \item \textbf{F.IF.9}: Compare and interpret functions represented algebraically, graphically, or in tables.
    \item \textbf{F.BF.1c}: Compose functions and interpret the composition in context.
    \item \textbf{MP.1, MP.2, MP.4}: Make sense of problems, reason abstractly, and model with mathematics.
\end{itemize}

%%%%%%%%%%%%%%%%%%%%%%%%%%%%%%%%%%%%%%%%%%%%%%%%%%%%%
% OBJECTIVES
%%%%%%%%%%%%%%%%%%%%%%%%%%%%%%%%%%%%%%%%%%%%%%%%%%%%%

\section*{Learning Objectives}
Students will be able to:
\begin{enumerate}[leftmargin=2em]
    \item Evaluate functions from formulas, tables, and graphs.
    \item Understand function composition as \textbf{doing one function after another}.
    \item Use visual arrow-diagrams to track how inputs move through functions.
    \item Compute compositions in algebraic, tabular, and graphical settings.
    \item Apply composition to model contextual problems.
\end{enumerate}

%%%%%%%%%%%%%%%%%%%%%%%%%%%%%%%%%%%%%%%%%%%%%%%%%%%%%
% MATERIALS
%%%%%%%%%%%%%%%%%%%%%%%%%%%%%%%%%%%%%%%%%%%%%%%%%%%%%

\section*{Materials}
\begin{itemize}[leftmargin=2em]
    \item Function Evaluation and Composition Reference Sheet (included)
    \item Worksheet with scaffolding (included)
    \item Whiteboards or scrap paper for practice
    \item Graphing tools (optional)
\end{itemize}

%%%%%%%%%%%%%%%%%%%%%%%%%%%%%%%%%%%%%%%%%%%%%%%%%%%%%
% PROCEDURE
%%%%%%%%%%%%%%%%%%%%%%%%%%%%%%%%%%%%%%%%%%%%%%%%%%%%%

\section*{Lesson Procedure}

\subsection*{1. Launch (5 minutes)}
Ask students:
\begin{quote}
“What does it mean to \emph{evaluate} a function? What does it mean to \emph{compose} two functions?”
\end{quote}

Address misconceptions:
\begin{itemize}
    \item Students often think composition is multiplication.
    \item Students mix up inside/outside functions.
    \item Students confuse evaluating formulas, tables, and graphs.
\end{itemize}

Explain that today we will rebuild the foundation using clear, simple steps.

\subsection*{2. Rebuilding the Foundation (10 minutes)}
Students read the provided \textbf{Reference Sheet}, which:
\begin{itemize}
    \item reviews evaluating functions (formula, table, graph),
    \item shows composition as a sequence of inputs flowing through boxes,
    \item includes visual examples.
\end{itemize}

Teacher models one example of each type:
\begin{itemize}
    \item Numeric evaluation
    \item Table lookup
    \item Reading a graph
    \item Simple composition using arrow-diagram
\end{itemize}

\subsection*{3. Guided Practice (10--12 minutes)}
Work through 1–2 sample problems as a class using the arrow-diagram format.  
Emphasize the step-by-step flow:
\[
x \rightarrow f(x) \rightarrow g(f(x))
\]

\subsection*{4. Independent Practice (15 minutes)}
Students complete the worksheet, progressing from:
\begin{enumerate}
    \item algebraic evaluation and composition,
    \item composition using tables,
    \item composition using graphs,
    \item contextual modeling with composition.
\end{enumerate}

Teacher circulates to support students who struggle with:
\begin{itemize}
    \item choosing the correct table column,
    \item identifying the correct $x$-coordinate on a graph,
    \item thinking “inside first” when composing.
\end{itemize}

\subsection*{5. Summary (5 minutes)}
Ask exit reflection questions:
\begin{itemize}
    \item “When I see $g(f(3))$, what is the very first thing I should do?”
    \item “How do I evaluate a function from a graph?”
    \item “What does composition represent in a real-world context?”
\end{itemize}
\end{document}
