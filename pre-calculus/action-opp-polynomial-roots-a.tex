\documentclass[10pt]{article}

\usepackage{geometry}
\geometry{margin=3em,top=6em, left=2cm, headheight=\paperheight}
\usepackage[export]{adjustbox}
\usepackage{array}
\usepackage{amsmath}
\usepackage{amsfonts}
\usepackage{fancyhdr}
\pagestyle{fancy}
\fancyhf{}
\lhead{Pre-Calculus}
\chead{Polynomial Roots}
\rhead{Action Opportunity A, Page \thepage}
\usepackage{lastpage}
\usepackage{xcolor}
\usepackage{enumitem}
\usepackage{pifont}
\usepackage{graphicx}
\graphicspath{{../img}}
\usepackage{pgfplots}
\pgfplotsset{compat=1.18}
\usepackage{tabularx}

\newcommand{\R}{\mathbb R}
\newcommand{\e}{{\rm e}}
\newcommand{\pobr}[1]{\left\langle#1\right\rangle}
\newcommand{\norm}[1]{\lVert #1 \rVert}
\newcommand{\abs}[1]{\lvert #1 \rvert}

\DeclareMathOperator{\xd}{d\!}
\DeclareMathOperator{\proj}{proj}

\title{}
\date{}

\begin{document}
\noindent
{\large
Name \rule{16em}{.5pt}\hspace{\stretch{1}} Date \rule{8em}{.5pt}\hspace{\stretch{1}} Period \rule{2em}{.5pt}
}
\vspace{1em}

\begin{enumerate}
    \item Complete the definition: $c$ is a root of the polynomial $p(x)$ if \rule{10em}{0.5pt} .
    \item Factorize the quadratic polynomial \(x^2 + 4x + 3\).
    \vspace{\stretch{1}}
    \item Factorize the quadratic polynomial \(\displaystyle x^2 - \frac{7x}{20}- \frac{3}{10}\). (Fact: \(\sqrt{529} = 23\), \(\sqrt{400} = 20\))
    \vspace{\stretch{1}}
\end{enumerate}

\end{document}