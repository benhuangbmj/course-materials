\documentclass[10pt]{article}

\usepackage{geometry}
\geometry{margin=3em,top=6em, left=2.5cm, headheight=\paperheight}
\usepackage[export]{adjustbox}
\usepackage{array}
\usepackage{amsmath}
\usepackage{amsfonts}
\usepackage{fancyhdr}
\pagestyle{fancy}
\fancyhf{}
\lhead{Pre-calc}
\chead{Polynomial Long Division}
\rhead{Action Opportunity A, Page \thepage}
\usepackage{lastpage}
\usepackage{xcolor}
\usepackage{enumitem}
\usepackage{pifont}
\usepackage{graphicx}
\graphicspath{{../img}}
\usepackage{pgfplots}
\pgfplotsset{compat=1.18}
\usepackage{tabularx}

\newcommand{\R}{\mathbb R}
\newcommand{\e}{{\rm e}}
\newcommand{\pobr}[1]{\left\langle#1\right\rangle}
\newcommand{\norm}[1]{\lVert #1 \rVert}
\newcommand{\abs}[1]{\lvert #1 \rvert}

\DeclareMathOperator{\xd}{d\!}
\DeclareMathOperator{\proj}{proj}

\title{}
\date{}

\begin{document}
\noindent
{
Name \rule{16em}{.5pt}\hspace{\stretch{1}} Date \rule{8em}{.5pt}\hspace{\stretch{1}} Period \rule{4em}{.5pt}\hspace{\stretch{1}} Grade  \rule{4em}{.5pt}
}
\vspace{1em}

{\noindent\bf Problems.}

\begin{enumerate}
    \item State the relation between the roots and the factors of a polynomial: if $c$ is a root of the polynomial \(p(x)\), then\\[1em] \(\displaystyle p(x) =\ \rule{40em}{0.5pt}\)
    \item Given 5 is a root of \(p(x) =x^3 + 5 x^2 - 34 x - 80\), find all the roots of \(p(x)\).
\end{enumerate}


\end{document}