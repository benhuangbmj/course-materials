\noindent
{\large
First Name \rule{6em}{.1pt}\hspace{\stretch{1}} Last Name \rule{6em}{.1pt}\hspace{\stretch{1}} Date \rule{1.5em}{.1pt} -- \rule{1.5em}{.1pt} -- \rule{1.5em}{.1pt}\hspace{\stretch{1}} Period \rule{2em}{.1pt}\hspace{\stretch{1}} Score \rule{2em}{.1pt}
}
\vspace{1em}

Let $f(x)$ be a function. The slope of the tangent line of $f(x)$ at an arbitrary point $x$ is 
\[
f'(x) = \lim_{\Delta x\to 0}\frac{f(x + \Delta x) - f(x)}{\Delta x}
\]
This concept is typically called the \textbf{derivative} of $f(x)$ in mathematics literature, denoted by $f'(x)$ (read ``f prime of x''.) The \textbf{gradient} of a single variable function (i.e., the variable is one dimensional) is effectively just the derivative.

Consider the function
\begin{luacode*}
a = genCoeff(-10,10,false,false,true)
tex.print("\\polymul\\function{x-(".. a .. ")}{(x-(" .. tonumber(a) + 3 .. "))(x-(" .. tonumber(a) + 8 .. "))}")
tex.print("\\[f(x) = \\polyprint\\function\\]")
\end{luacode*}
We want to find its minimum via the \textbf{gradient descent method}. In this lesson, we will use GeoGebra to assist the calculation.
\begin{enumerate}
\item
Plot the graph of $f(x)$ in GeoGebra.
\item
Find the derivative of $f(x)$ with the ``Derivative'' command in GeoGebra.
\vspace{\stretch{1}}
\item 
Find the point on the graph with $x = \directlua{tex.print(tonumber(a) + 4)}$, then find the derivative at this point. (Remark: Be sure to plot the point in GeoGebra and check if it's really on the graph!)
\vspace{1em}
\begin{itemize}
\item
Point: \rule{8em}{.4pt}\\[1em]
\item 
Derivative \rule{8em}{.4pt}
\end{itemize}
\item Find an equation of the tangent line of $f(x)$ at the point you found, then
\begin{itemize}
\item
Plot the equation that you found in GeoGebra.
\item
Plot the tangent line at the point in GeoGebra with the built-in feature. Do the two lines match?
\vspace{\stretch{1}}
\clearpage

\end{itemize}
\item Now we start to use the gradient descent to find the minimum of $f(x)$. Start from the point you found, then (Remark: Feel free to use the code template in Google Classroom and adapt.)
\begin{itemize}
\item
Try three different descent steps: 0.1, 0.01, and 0.001.
\vspace{\stretch{1}}
\item
Plot and compare the three results in GeoGebra.
\vspace{\stretch{1}}
\item Count how many steps you have taken to get the results, respectively. 
\vspace{\stretch{1}}
\end{itemize}
\end{enumerate}
\clearpage