\documentclass[twoside, 10pt]{article}

\usepackage{geometry}
\geometry{outer=3em, inner=2.2cm, top=6em, bottom=4em, headheight=\paperheight}
\usepackage[export]{adjustbox}
\usepackage{array}
\usepackage{amsmath}
\usepackage{amsfonts}
\usepackage{fancyhdr}
\pagestyle{fancy}
\fancyhf{}
\lhead{Precalculus - BASE}
\chead{Limit 2}
\rhead{AO Make, Page \thepage}
\usepackage{lastpage}
\usepackage{xcolor}
\usepackage{enumitem}
\usepackage{pifont}
\usepackage{graphicx}
\graphicspath{{../img}}
\usepackage{pgfplots}
\pgfplotsset{compat=1.18}
\usepackage{tabularx}
\usepackage{tikz}
\usetikzlibrary{patterns}

\newcommand{\R}{\mathbb R}
\newcommand{\e}{{\rm e}}
\newcommand{\pobr}[1]{\left\langle#1\right\rangle}
\newcommand{\norm}[1]{\lVert #1 \rVert}
\newcommand{\abs}[1]{\lvert #1 \rvert}

\DeclareMathOperator{\xd}{d\!}
\DeclareMathOperator{\proj}{proj}

\title{}
\date{}

\begin{document}
\noindent
{\large
First Name \rule{6em}{.1pt}\hspace{\stretch{1}}Last Name \rule{6em}{.1pt}\hspace{\stretch{1}} Due \rule{1.5em}{.1pt} -- \rule{1.5em}{.1pt} -- \rule{1.5em}{.1pt}\hspace{\stretch{1}} Period \rule{2em}{.1pt}\hspace{\stretch{1}} Score \rule{2em}{.1pt}
}
\vspace{1em}

\begingroup
\renewcommand{\arraystretch}{1.5}
\begin{center}
\tiny
{
\begin{tabularx}{\textwidth}{|X|X|X|X|X|X|}
\hline
\bf MODEL & \centerline{Integrating} & \centerline{Applying} & \centerline{Practicing} & \centerline{Acquiring} & \centerline{Awaiting Evidence} \\
\hline
I can use math to model and solve real-world problems.&
Correctly identifies
important
quantities and
illustrates their
relationships using
diagrams, tables,
graphs, or
formulas.
Appropriate work is
shown with no
errors. The answer
includes units and
rounding as
appropriate to the
problem.
Explains how the
answer makes
sense in the
context of the
problem.
&Correctly identifies
important
quantities and
illustrates their
relationships using
diagrams, tables,
graphs, or
formulas.
Appropriate work is
shown with no
errors. The answer
includes units and
rounding as
appropriate to the
problem.
&Correctly identifies
important
quantities and
illustrates their
relationships using
diagrams, tables,
graphs, or
formulas.
Appropriate work is
shown with 1
COMPUTATIONAL
or ROUNDING
error.
&Correctly identifies
important
quantities and
attempts to
illustrate their
relationships using
diagrams, tables,
graphs, or formulas
Appropriate work is
shown with 1
CONCEPTUAL
error.
&Correctly identifies
important
quantities and
attempts to
illustrate their
relationships using
diagrams, tables,
graphs, or formulas
Appropriate work is
shown with more
than 1 conceptual
error.\\
\hline
\bf Criteria&\multicolumn{5}{l|}{\parbox[c][4em]{.8\textwidth}{}}\\
\hline
\end{tabularx}
}
\end{center}
\endgroup
\vspace{.5em}

\begin{enumerate}
\item Create a problem of calculating a limit meeting the following requirements, then solve it.
\begin{enumerate}
\item
It finds the limit of a function at the number $n=\text{an integer}$. 
\item
It's a fraction and when plugging in $n$ directly it becomes ``$\displaystyle \frac{0}{0}$''.
\item
It has a square root expression at either the numerator or the denominator.
\item
It's solvable with the ``multiplying the conjugate'' trick.
\end{enumerate}
Here is a model problem: $\displaystyle \lim_{x\to2}\frac{\sqrt{x+2} - 2}{x-2}$
\vspace{\stretch{1}}
\item Create a problem of calculating a limit at the positive infinity meeting the following requirements, then solve it.
\begin{itemize}
\item
The function is a rational function, namely, a quotient between two polynomials.
\item 
The answer is a nonzero number.
\end{itemize}
\vspace{\stretch{1}}
\item Create a problem of calculating a limit at the negative infinity meeting the following requirements, then solve it.
\begin{itemize}
\item
The function is a rational function, namely, a quotient between two polynomials.
\item 
The answer is the positive infinity.
\end{itemize}
\vspace{\stretch{1}}
\end{enumerate}

\end{document}