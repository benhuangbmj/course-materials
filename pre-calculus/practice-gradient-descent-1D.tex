\documentclass[twoside, 10pt]{article}

\usepackage{geometry}
\geometry{bottom=4em,top=6em, inner=2.2cm, outer=4em,  headheight=\paperheight}
\usepackage[export]{adjustbox}
\usepackage{array}
\usepackage{amsmath}
\usepackage{amsfonts}
\usepackage{fancyhdr}
\pagestyle{fancy}
\fancyhf{}
\lhead{Precalculus - BASE}
\chead{Gradient Descent - 1D}
\rhead{Practice, Page \thepage}
\usepackage{lastpage}
\usepackage{xcolor}
\usepackage{enumitem}
\usepackage{pifont}
\usepackage{graphicx}
\graphicspath{{../img}}
\usepackage{pgfplots}
\pgfplotsset{compat=1.18}
\usepackage{tabularx}
\usepackage{luacode}

\newcommand{\R}{\mathbb R}
\newcommand{\e}{{\rm e}}
\newcommand{\pobr}[1]{\left\langle#1\right\rangle}
\newcommand{\norm}[1]{\lVert #1 \rVert}
\newcommand{\abs}[1]{\lvert #1 \rvert}

\DeclareMathOperator{\xd}{d\!}
\DeclareMathOperator{\proj}{proj}

\title{}
\date{}

\begin{luacode*}
utils = require("../utils")
genCoeff = utils.genRandom
\end{luacode*}

\begin{document}
\noindent
{\large
First Name \rule{6em}{.1pt}\hspace{\stretch{1}}Last Name \rule{6em}{.1pt}\hspace{\stretch{1}} {Date} \rule{1.5em}{.1pt} -- \rule{1.5em}{.1pt} -- \rule{1.5em}{.1pt}\hspace{\stretch{1}} Period \rule{2em}{.1pt}\hspace{\stretch{1}} Score \rule{2em}{.1pt}
}
\vspace{1em}

Let $f(x)$ be a function. The slope of the tangent line of $f(x)$ at an arbitrary point $x$ is 
\[
f'(x) = \lim_{\Delta x\to 0}\frac{f(x + \Delta x) - f(x)}{\Delta x}
\]
This concept is typically called the \textbf{derivative} of $f(x)$ in mathematics literature, denoted by $f'(x)$ (read ``f prime of x''.) The \textbf{gradient} of a single variable function (i.e., the variable is one dimensional) is effectively just the derivative.

Consider the function \[f(x) = \directlua{tex.print(genCoeff(-1,1, false, true, true))}x^3 \directlua{tex.print(genCoeff(-3,3, true, true, true))}x^2 \directlua{tex.print(genCoeff(-10,10, true, true, true))}x  \directlua{tex.print(genCoeff(-10,10, true, false, true))}.\] We want to find its minimum via the \textbf{gradient descent method}. In this lesson, we will use GeoGebra to assist the calculation.

\begin{enumerate}
\item
Plot the graph of $f(x)$ in GeoGebra.
\item
Find the derivative of $f(x)$.
\vspace{\stretch{1}}
\item 
Pick a point on the graph, then find the derivative at this point. (Remark: Be sure to plot the point in GeoGebra and check if it's really on the graph!)
\vspace{1em}
\begin{itemize}
\item
Your point: \rule{8em}{.4pt}\\[1em]
\item 
Your derivative \rule{8em}{.4pt}
\end{itemize}
\item Find an equation of the tangent line of $f(x)$ at the point you picked, then
\begin{itemize}
\item
Plot the equation that you found in GeoGebra.
\item
Plot the tangent line at your point in GeoGebra with the built-in feature. Do the two lines match?
\vspace{\stretch{1}}
\clearpage

\end{itemize}
\item Now we start to use the gradient descent to find the minimum of $f(x)$. Start from the point you chose, then (Remark: Feel free to use the code template in Google Classroom and adapt.)
\begin{itemize}
\item
Try three different descent steps: 0.1, 0.01, and 0.001.
\vspace{\stretch{1}}
\item
Plot and compare the three results in GeoGebra.
\vspace{\stretch{1}}
\item Count how many steps you have taken to get the results, respectively. 
\vspace{\stretch{1}}
\end{itemize}
\end{enumerate}


\end{document}