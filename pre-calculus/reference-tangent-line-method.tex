\documentclass[twoside, 10pt]{article}

\usepackage{geometry}
\geometry{bottom=4em,top=6em, inner=2.2cm, outer=4em,  headheight=\paperheight}
\usepackage[export]{adjustbox}
\usepackage{array}
\usepackage{amsmath}
\usepackage{amsfonts}
\usepackage{fancyhdr}
\pagestyle{fancy}
\fancyhf{}
\lhead{Precalculus - BASE}
\chead{Tangent Line Equation through Limit}
\rhead{Reference, Page \thepage}
\usepackage{lastpage}
\usepackage{xcolor}
\usepackage{enumitem}
\usepackage{pifont}
\usepackage{graphicx}
\graphicspath{{../img}}
\usepackage{pgfplots}
\pgfplotsset{compat=1.18}
\usepackage{tabularx}
\usepackage{polynom}

\newcommand{\R}{\mathbb R}
\newcommand{\e}{{\rm e}}
\newcommand{\pobr}[1]{\left\langle#1\right\rangle}
\newcommand{\norm}[1]{\lVert #1 \rVert}
\newcommand{\abs}[1]{\lvert #1 \rvert}

\DeclareMathOperator{\xd}{d\!}
\DeclareMathOperator{\proj}{proj}

\title{}
\date{}

\begin{document}
\noindent
{\large
First Name \rule{6em}{.1pt}\hspace{\stretch{1}} Last Name \rule{6em}{.1pt}\hspace{\stretch{1}} {Date} \rule{1.5em}{.1pt} -- \rule{1.5em}{.1pt} -- \rule{1.5em}{.1pt}\hspace{\stretch{1}} Period \rule{2em}{.1pt}\hspace{\stretch{1}} Score \rule{2em}{.1pt}
}
\vspace{1em}

\section*{Goal}
Find the tangent line to the graph of a function \( f(x) \) at the point where \( x = a \).  
This method uses the limit of secant slopes and does \emph{not} reference the advanced concept \textit{derivatives}.

\vspace{1em}

\section*{Step 1: Identify the Fixed Point}
Compute the point on the graph:
\[
(a,\; f(a)).
\]

This is the point where you will draw the tangent line.

\vspace{1em}

\section*{Step 2: Write the Slope of a Secant Line}
Choose another point on the graph with \( x \neq a \).  
That point is \( (x, f(x)) \).

The slope of the secant line between \( (a, f(a)) \) and \( (x, f(x)) \) is
\[
m_{\text{secant}}(x)
= \frac{f(x) - f(a)}{x - a}.
\]

This expression is a \emph{function of \( x \)}.

\vspace{1em}

\section*{Step 3: Move the Second Point Toward the First}
To get the slope of the tangent line, let the second point slide toward the fixed point:
\[
m_{\text{tangent}}
= \lim_{x \to a} \frac{f(x) - f(a)}{x - a}.
\]

If this limit exists, it gives the \emph{exact} slope of the tangent line at \( x = a \).

\vspace{1em}

\section*{Step 4: Evaluate the Limit}
Simplify
\[
\frac{f(x) - f(a)}{x - a}
\]
as much as possible.  
Then substitute \( x = a \) \emph{after simplifying}.  
The result is the slope of the tangent line.

\vspace{1em}

\section*{Step 5: Write the Equation of the Tangent Line}
Use point--slope form with the point \( (a, f(a)) \) and the slope \( m_{\text{tangent}} \):
\[
y - f(a) = m_{\text{tangent}}(x - a).
\]

You may convert to slope--intercept form if needed.

\vspace{2em}

\section*{Summary}
\begin{enumerate}
    \item Compute \( f(a) \).
    \item Form the secant slope function:
    \[
    m_{\text{secant}}(x) = \frac{f(x) - f(a)}{x - a}.
    \]
    \item Simplify the expression.
    \item Take the limit as \( x \to a \):
    \[
    m_{\text{tangent}} = \lim_{x \to a} m_{\text{secant}}(x).
    \]
    \item Write the tangent line:
    \[
    y = f(a) + m_{\text{tangent}}(x - a).
    \]
\end{enumerate}

\end{document}