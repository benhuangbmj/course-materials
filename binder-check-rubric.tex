\begin{center}
    {\huge \textbf{Binder Check Rubric}}
\end{center}

\vspace{1.5em}

\noindent
The purpose of the binder check is to ensure that you maintain an organized and complete record of your class materials. 
Follow the expectations below carefully. Points may be deducted for missing items, incomplete work, or disorderly presentation.

\section*{Student Responsibilities}

\begin{enumerate}[leftmargin=*]

    \item \textbf{Checklist Preparation.}\\
    Before the binder check, you will receive a checklist of required materials for the current period.  
    You must keep these materials organized as we progress through the class.

    \vspace{0.7em}
    \textbf{Sample Checklist (Mock):}
    \begin{enumerate}[leftmargin=*, label=\raisebox{-.4ex}{\scalebox{1.5}{$\square$}}\ \arabic*.]
        \item Warm-Up Sheets (Weeks 5–6)
        \item Notes: Solving Linear Equations
        \item Notes: Systems of Equations
        \item Classwork: Linear Equations Practice
        \item Homework Set \#4
        \item Quiz Corrections (if applicable)
        \item Reflection Sheet (end-of-unit)
    \end{enumerate}

    \item \textbf{Material Organization.}\\
    All materials must be placed in the binder \emph{in the exact order} listed on the checklist.

    \item \textbf{Completion of Assignments.}\\
    All exercises, problem sets, notes, and worksheets included in the binder must be fully completed.

    \item \textbf{Self-Verification.}\\
    You must check off each item on the checklist to confirm that:
    \begin{itemize}
        \item the item is present,
        \item the item is in the correct order,
        \item the work on that item is fully complete.
    \end{itemize}

    \item \textbf{Checklist Attachment.}\\
    The completed checklist must be attached at the \emph{front} of the binder as the first page.

\end{enumerate}




