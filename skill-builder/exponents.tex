\documentclass[twoside, 10pt]{article}

\usepackage{geometry}
\geometry{outer=3em, inner=2.2cm, top=6em, bottom=4em, headheight=\paperheight}
\usepackage[export]{adjustbox}
\usepackage{array}
\usepackage{amsmath}
\usepackage{amsfonts}
\usepackage{fancyhdr}
\pagestyle{fancy}
\fancyhf{}
\lhead{Skill Builder}
\chead{Exponents and Exponent Rules}
\rhead{Rebuild, Page \thepage}
\usepackage{lastpage}
\usepackage{xcolor}
\usepackage{enumitem}
\usepackage{pifont}
\usepackage{graphicx}
\graphicspath{{../img}}
\usepackage{pgfplots}
\pgfplotsset{compat=1.18}
\usepackage{tabularx}
\usepackage{tikz}
\usetikzlibrary{patterns}
%\usepackage{luacode}

\newcommand{\R}{\mathbb R}
\newcommand{\e}{{\rm e}}
\newcommand{\pobr}[1]{\left\langle#1\right\rangle}
\newcommand{\norm}[1]{\lVert #1 \rVert}
\newcommand{\abs}[1]{\lvert #1 \rvert}

\DeclareMathOperator{\xd}{d\!}
\DeclareMathOperator{\proj}{proj}

\title{Review of Exponents and Exponent Rules}
\author{}
\date{}

\begin{document}
\section*{1. What is an Exponent?}
Let $n$ be a natural number, namely, a positive integer.
An expression of the form
\[
a^n
\]
means that the base $a$ is multiplied by itself $n$ times. That is,
\[
a^n = \underbrace{a \cdot a \cdot a \cdots a}_{n \text{ times}}.
\]

\subsection*{Examples}
\[
2^4 = 2\times2\times2\times2 = 16, \qquad 5^1 = 5
\]

\section*{2. Fundamental Exponent Rules}

\subsection*{Rule 1: Product of Powers}
If the bases are the same, add the exponents:
\[
a^m \cdot a^n = a^{m+n}.
\]

\textit{Example:}
\[
x^3 \cdot x^5 = (x\cdot x\cdot x)(x\cdot x\cdot x\cdot x\cdot x) = x\cdot x\cdot x\cdot x\cdot x\cdot x\cdot x\cdot x= x^8 = x^{3+5}.
\]

\subsection*{Rule 2: Quotient of Powers}
If the bases are the same, subtract the exponents:
\[
\frac{a^m}{a^n} = a^{m-n} \quad (a \neq 0).
\]

\textit{Example:}
\[
\frac{y^4}{y^2} = \frac{y\cdot y\cdot y\cdot y}{y\cdot y} = \frac{y\cdot y}{1} = y^2 =	y^{4-2}.
\]

\subsection*{Rule 3: Power of a Power}
Multiply the exponents:
\[
(a^m)^n = a^{mn}.
\]

\textit{Example:}
\[
(x^2)^3 = (x^2) (x^2) (x^2) = (x\cdot x) (x\cdot x) (x\cdot x) = x\cdot x\cdot x\cdot x\cdot x\cdot x= x^6 = x^{2\times 3}.
\]

\subsection*{Rule 4: Power of a Product}
\[
(ab)^n = a^n b^n.
\]

\textit{Example:} 
\[
(xy)^3 = (xy)(xy)(xy) = (xxx)(yyy) = x^3y^3.
\]

\subsection*{Rule 5: Power of a Quotient}
\[
\left( \frac{a}{b} \right)^n = \frac{a^n}{b^n} \quad (b \neq 0).
\]

\textit{Example:}
\[
\left(\frac{2x}{3y}\right)^3 = \frac{2^3 x^3}{3^3 y^3} = \frac{8x^3}{27y^3}.
\]

\section*{3. Zero and Negative Exponents}

\subsection*{Zero Exponent}
According to \textbf{Rule 2}, \textit{Quotient of Powers},
\[
a^0 = a^{1-1} = \frac{a^1}{a^1} = \frac{a}{a} =  1 \quad (a \neq 0).
\]

\textit{Examples:}
\[
7^0 = 1, \qquad (-3)^0 = 1.
\]

\subsection*{Negative Exponent}
\[
a^{-n} = a^{0-n} = \frac{a^0}{a^n} = \frac{1}{a^n} \quad (a \neq 0).
\]

\textit{Examples:}
\[
x^{-3} = \frac{1}{x^3}, 
\qquad 
10^{-2} = \frac{1}{10^2} = \frac{1}{100}.
\]

\section*{4. Fractional Exponents}
According to \textbf{Rule 3}, \textit{Power of a Power},
\[
\left(a^\frac{1}{n}\right)^n = a^{\frac{1}{n}\cdot n} = a^1 = a.
\]
Therefore, a fractional exponent represents roots:
\[
a^{\frac{1}{n}} = \sqrt[n]{a},
\qquad
a^{\frac{m}{n}} = \sqrt[n]{a^m} = \left(\sqrt[n]{a}\right)^m.
\]

\textit{Examples:}
\[
16^{1/2} = 4, 
\qquad 
27^{2/3} = \left( \sqrt[3]{27} \right)^2 = 3^2 = 9.
\]

\section*{5. Practice Problems}

\begin{enumerate}
  \item Simplify: $x^5 \cdot x^7$\quad\rule{10em}{.1pt} .
  \vspace{\stretch{1}}
  \item Simplify: $\dfrac{a^{12}}{a^3}$\quad\rule{10em}{.1pt} .
  \vspace{\stretch{1}}
  \item Simplify: $(y^4)^3$\quad\rule{10em}{.1pt} .
  \vspace{\stretch{1}}
  \item Rewrite with positive exponents only: $b^{-6}$\quad\rule{10em}{.1pt} .\\
  (Hint: Use a fraction)
  \vspace{\stretch{1}}
  \item Evaluate without a calculator: $32^{1/5}$\quad\rule{10em}{.1pt} .
  \vspace{\stretch{1}}
  \item Simplify completely: $\left( \dfrac{3x^2}{y^3} \right)^2$\quad\rule{10em}{.1pt} .
  \vspace{\stretch{1}}
\end{enumerate}
\section*{6. Exit Tickets}
An exponential function is of the form $f(x) = ab^x$. Given $f(0) = 25$ and $f(1) = 20$, find $f(x)$. \quad\rule{10em}{.1pt} .

\end{document}