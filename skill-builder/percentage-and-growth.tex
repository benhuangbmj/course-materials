\documentclass[twoside, 10pt]{article}

\usepackage{geometry}
\geometry{outer=3em, inner=2.2cm, top=6em, bottom=4em, headheight=\paperheight}
\usepackage[export]{adjustbox}
\usepackage{array}
\usepackage{amsmath}
\usepackage{amsfonts}
\usepackage{fancyhdr}
\pagestyle{fancy}
\fancyhf{}
\lhead{Skill Builder}
\chead{Rational Exponent}
\rhead{Page \thepage}
\usepackage{lastpage}
\usepackage{xcolor}
\usepackage{enumitem}
\usepackage{pifont}
\usepackage{graphicx}
\graphicspath{{../img}}
\usepackage{pgfplots}
\pgfplotsset{compat=1.18}
\usepackage{tabularx}
\usepackage{tikz}
\usetikzlibrary{patterns}
\usepackage{luacode}

\newcommand{\R}{\mathbb R}
\newcommand{\e}{{\rm e}}
\newcommand{\pobr}[1]{\left\langle#1\right\rangle}
\newcommand{\norm}[1]{\lVert #1 \rVert}
\newcommand{\abs}[1]{\lvert #1 \rvert}

\DeclareMathOperator{\xd}{d\!}
\DeclareMathOperator{\proj}{proj}

\title{}
\date{}

\begin{document}
\noindent
{\large
First Name \rule{6em}{.1pt}\hspace{\stretch{1}}Last Name \rule{6em}{.1pt}\hspace{\stretch{1}} Date \rule{1.5em}{.1pt} -- \rule{1.5em}{.1pt} -- \rule{1.5em}{.1pt}\hspace{\stretch{1}} Period \rule{2em}{.1pt}\hspace{\stretch{1}} Score \rule{2em}{.1pt}
}
\vspace{1em}

\begin{luacode*}
math.randomseed(os.time());
a1 = math.random(500,2000)
b1 = math.random(1,15)
c1 = (a1*b1)/100
expr1 = string.format("%d(%.2f) = %.2f", a1,b1/100, (a1*b1)/100)
expr2 = string.format("%d(1 + %.2f) = %.2f", a1,b1/100, a1*(1+b1/100))
\end{luacode*}

\noindent{\bf Review of Concepts}
\begin{itemize}
\item
{\bf Percentage.} $\displaystyle p\% = \frac{p}{100}$. The $p\%$ of the quantity $Q$ is computed by $\displaystyle Q\left(\frac{p}{100}\right)$. If the quantity $Q$ grows by $p\%$, the final quantity will be 
\[
Q + Q\left(\frac{p}{100}\right) = Q\left(1 + \frac{p}{100}\right).
\]
For example, the $\directlua{tex.print(b1)}\%$ of $\directlua{tex.print(a1)}$ dollars is $\directlua{tex.print(expr1)}$ dollars, and if your investment of $\directlua{tex.print(a1)}$ dollars grows by  $\directlua{tex.print(b1)}\%$ in the first year, you will end up with $\directlua{tex.print(expr2)}$ dollars.

If the quantity $Q$ stably grows by $p\%$ again and again, then you have
\[
Q,\ Q\left(1 + \frac{p}{100}\right),\ \left(Q\left(1 + \frac{p}{100}\right)\right)\left(1 + \frac{p}{100}\right) = Q\left(1 + \frac{p}{100}\right)^2,\ Q\left(1 + \frac{p}{100}\right)^3,\ \dots
\]
\end{itemize}

\noindent{\bf Exercises.}

\end{document}