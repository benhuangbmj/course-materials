\documentclass[twoside, 10pt]{article}

\usepackage{geometry}
\geometry{outer=3em, inner=2.2cm, top=6em, bottom=4em, headheight=\paperheight}
\usepackage[export]{adjustbox}
\usepackage{array}
\usepackage{amsmath}
\usepackage{amsfonts}
\usepackage{fancyhdr}
\pagestyle{fancy}
\fancyhf{}
\lhead{Skill Builder}
\chead{Positive Rational Exponents}
\rhead{Rebuild, Page \thepage}
\usepackage{lastpage}
\usepackage{xcolor}
\usepackage{enumitem}
\usepackage{pifont}
\usepackage{graphicx}
\graphicspath{{../img}}
\usepackage{pgfplots}
\pgfplotsset{compat=1.18}
\usepackage{tabularx}
\usepackage{tikz}
\usetikzlibrary{patterns}
%\usepackage{luacode}

\newcommand{\R}{\mathbb R}
\newcommand{\e}{{\rm e}}
\newcommand{\pobr}[1]{\left\langle#1\right\rangle}
\newcommand{\norm}[1]{\lVert #1 \rVert}
\newcommand{\abs}[1]{\lvert #1 \rvert}

\DeclareMathOperator{\xd}{d\!}
\DeclareMathOperator{\proj}{proj}

\title{Review of Exponents and Exponent Rules}
\author{}
\date{}

\begin{document}
\noindent
{\large
First Name \rule{6em}{.1pt}\hspace{\stretch{1}} Last Name \rule{6em}{.1pt}\hspace{\stretch{1}} {Date} \rule{1.5em}{.1pt} -- \rule{1.5em}{.1pt} -- \rule{1.5em}{.1pt}\hspace{\stretch{1}} Period \rule{2em}{.1pt}\hspace{\stretch{1}} Score \rule{2em}{.1pt}
}
\vspace{1em}

\noindent
{\bf Exercises.}
\begin{enumerate}
\item A number can be written in many different ways. For example,
\[
	\frac{3}{25} = 3 \cdot \frac{1}{25} = \frac{1}{25} \cdot 3 = \frac{3}{5}\cdot \frac{1}{5}
\]
Can you solve the following puzzles?
\begin{enumerate}
\item
$\displaystyle \frac{2}{3} = \frac{1}{3}(\rule{2em}{0.4pt}) = 2 (\rule{2em}{0.4pt})$\\[1em]
\item
$\displaystyle \frac{3}{10} = \frac{1}{10} (\rule{2em}{0.4pt})  = 3 (\rule{2em}{0.4pt})  = \frac{3}{2} (\rule{2em}{0.4pt})$\\[1em]
\item
$\displaystyle 5^{\frac{2}{3}} = (5^\frac{1}{3})^{(\rule{2em}{0.4pt})} = (5^2)^{(\rule{2em}{0.4pt})}$ (Hint: Can you apply the rule $a^{mn} = (a^m)^n $?)
\end{enumerate}
\item We just learned that radicals and unit fraction exponents are interchangeable. For example, 
\[
	5^\frac{1}{2} = \sqrt 5,\quad 5^\frac{1}{3} = \sqrt[3]5, \quad etc. 
\]
In fact, it's straightforward to extend the idea to any \textbf{rational number exponent}\footnote{recall that a rational number is just a fraction of integers, i.e. $\displaystyle \frac{m}{n}$, where $m$ and $n$ are integers and $n\ne 0$.}. Let's try to do this together:
\[
5^{\frac{2}{3}} = 
\]
\end{enumerate}
\end{document}