\documentclass[twoside, 10pt]{article}

\usepackage{geometry}
\geometry{outer=3em, inner=2.2cm, top=6em, bottom=4em, headheight=\paperheight}
\usepackage[export]{adjustbox}
\usepackage{array}
\usepackage{amsmath}
\usepackage{amsfonts}
\usepackage{fancyhdr}
\pagestyle{fancy}
\fancyhf{}
\lhead{Skill Builder}
\chead{Distributive Law and FOIL}
\rhead{Page \thepage}
\usepackage{lastpage}
\usepackage{xcolor}
\usepackage{enumitem}
\usepackage{pifont}
\usepackage{graphicx}
\graphicspath{{../img}}
\usepackage{pgfplots}
\pgfplotsset{compat=1.18}
\usepackage{tabularx}
\usepackage{tikz}
\usetikzlibrary{patterns}
\usepackage{luacode}

\newcommand{\R}{\mathbb R}
\newcommand{\e}{{\rm e}}
\newcommand{\pobr}[1]{\left\langle#1\right\rangle}
\newcommand{\norm}[1]{\lVert #1 \rVert}
\newcommand{\abs}[1]{\lvert #1 \rvert}

\DeclareMathOperator{\xd}{d\!}
\DeclareMathOperator{\proj}{proj}

\title{}
\date{}

\begin{document}
\noindent
{\large
First Name \rule{6em}{.1pt}\hspace{\stretch{1}}Last Name \rule{6em}{.1pt}\hspace{\stretch{1}} Date \rule{1.5em}{.1pt} -- \rule{1.5em}{.1pt} -- \rule{1.5em}{.1pt}\hspace{\stretch{1}} Period \rule{2em}{.1pt}\hspace{\stretch{1}} Score \rule{2em}{.1pt}
}
\vspace{1em}

\begin{luacode*}
math.randomseed(os.time());
a1 = math.random(-10, 10);
b1 = math.random(-10,10);
c1 = math.random(-10,10);
expr1 = string.format("&(%d) (%d %+d )\\\\[1em]=&(%d)(\\rule{3em}{.1pt})\\\\[1em]=&\\rule{4em}{.1pt}\\ .", a1, b1, c1, a1)
expr2 = string.format("&(%d)(%d) +( %d)(%d)\\\\[1em]=&\\rule{3em}{.1pt} + \\rule{3em}{.1pt}\\\\[1em]=&\\rule{4em}{.1pt}", a1,b1,a1,c1)

a2 = math.random(-99999, 99999)
b2 = math.random(-99999, 99999)
c2 = b2 - 1
a3 = math.random(-99999, 99999)
b3 = math.random(-99999, 99999)
c3 = b3 + 1
expr3 = string.format("&(%d)(%d) - (%d)(%d)\\\\[1em] =& (%d)((\\rule{4em}{.1pt}) - (\\rule{4em}{.1pt}))\\\\[1em] =& (%d)(\\rule{2em}{0.1pt})\\\\[1em] =& \\rule{4em}{.1pt}\\ .", a2, b2,a2,c2,a2,a2)
expr4 = string.format("(%d)(%d) + (%d)(%d))", a3, -b3, a3, c3)
\end{luacode*}

\noindent{\bf Distributive Law}: $a(b+c) = ab + ac$. \\
{\it Remark}:
\begin{itemize}
\item
Note that this implies $a(b-c) = ab - ac$, because $a(b-c) = a(b+(-c)) = ab + a(-c) = ab - ac$.
 \item 
It also implies that $a(b+c + d) = ab+ac+ad$, because
\[
a(b+c+d) = a((b+c)+d) = a(b+c) + ad = ab+ac+ad.
\]
\end{itemize}
\noindent{\bf Problems}
\begin{enumerate}
\item
Verify the distributive law for the following numbers:

\parbox{.45\textwidth}{
\begin{align*}
\directlua{tex.print(expr1)}
\end{align*}
}\parbox{.45\textwidth}{
\begin{align*}
\directlua{tex.print(expr2)}
\end{align*}
}
\item
Use the distributive law to solve the following problems:
\begin{enumerate}
\item
\begin{align*}
\directlua{tex.print(expr3)}
\end{align*}
\item $\directlua{tex.print(expr4)}$
\end{enumerate}
\end{enumerate}

\end{document}