\documentclass[10pt]{beamer}
\usetheme{PaloAlto}
\usecolortheme{seahorse}
\setbeamertemplate{navigation symbols}{}
\setbeamertemplate{caption}[numbered]
%general package
%\usepackage[utf8]{inputenc}
\usepackage[english]{babel}
\usepackage{geometry}
\usepackage{tcolorbox}
\usepackage[export]{adjustbox}
\usepackage{graphicx}
\graphicspath{{../img}}
\usepackage{graphbox}
%math package
\usepackage{amsmath}
\usepackage{amsfonts}
\usepackage{mathtools}
%\usepackage{amssymb}
%\usepackage{amsthm}
%\usepackage{slashed}
%\usepackage{tikz-cd}
%\usepackage{extarrows}
%font package
\usepackage{mathrsfs}
%\usepackage{bm}
%\usepackage{thmtools}
%misc. package
\usepackage{enumitem}
{
\author[B.H.]{\vspace{1em}\\{\Large MAT215 Intro to Linear Algebra}\\
\vspace{2em}
Instructor: Ben Huang}

\date{}
\title[]{Eigenvalues and Eigenvectors}
\institute[DCC]{\vskip -12em\includegraphics[width = 0.8\textwidth]{DCC.png}}
\logo{\includegraphics[width=0.12\textwidth]{DCC.png}}
%general package
\usepackage[utf8]{inputenc}
\usepackage[english]{babel}
\usepackage{geometry}

%math package
\usepackage{amsmath}
\usepackage{amsfonts}
\usepackage{amssymb}
\usepackage{amsthm}
\usepackage{slashed}
\usepackage{tikz-cd}

%font package
\usepackage{mathrsfs}
\usepackage{bm}

%misc. package
\usepackage{enumitem}
\usepackage{tcolorbox}
\usepackage{etoolbox}
\usepackage{hyperref}
\hypersetup{
  colorlinks=true, urlcolor=blue
}
%declared operators
\DeclareMathOperator{\id}{Id}%identity
\DeclareMathOperator{\ind}{Ind\!}%index
\DeclareMathOperator{\tr}{Tr}%trace
\DeclareMathOperator{\e}{e}%exponential
\DeclareMathOperator{\im}{Im\!}%image
\DeclareMathOperator{\vol}{vol}%volume
\DeclareMathOperator{\cll}{\C\ell}%complexified Clifford algebra
\DeclareMathOperator{\gd}{\slashed{\partial}}%geometric Dirac
\DeclareMathOperator{\D}{\mathcal{D}}%generalized Dirac
\DeclareMathOperator{\Div}{div}%divergence
\DeclareMathOperator{\ud}{\,\mathrm{d}\!}

\DeclareMathOperator{\Hom}{Hom}
\DeclareMathOperator{\xd}{\,d\!}
\DeclareMathOperator{\curl}{curl}
\DeclareMathOperator{\dive}{div}
\DeclareMathOperator{\proj}{proj}


\newcommand{\norm}[1]{\lVert#1\rVert}
\newcommand{\R}{\mathbb R}
\newcommand{\vF}{\mathbf F}
\newcommand{\vv}{\mathbf v}
\newcommand{\inpr}[1]{\left\langle#1\right\rangle}
\newcommand{\fix}{(a,b)}
\newcommand{\uv}{\mathbf u}
\newcommand{\abs}[1]{\lvert #1\rvert}
%texting in citation
\makeatletter
\let\cite\relax
\DeclareRobustCommand{\cite}{%
  \let\new@cite@pre\@gobble
  \@ifnextchar[\new@cite{\@citex[]}}
\def\new@cite[#1]{\@ifnextchar[{\new@citea{#1}}{\@citex[#1]}}
\def\new@citea#1{\def\new@cite@pre{#1}\@citex}
\def\@cite#1#2{[{\new@cite@pre\space#1\if\relax\detokenize{#2}\relax\else, #2\fi}]}
\makeatother

\begin{document}

\frame{\titlepage}

\begin{frame}
\frametitle{The Hierarchy of Matrices}
The simplest - scalar matrix:
\[
\begin{bmatrix}
c&0&0\\
0&c&0\\
0&0&c
\end{bmatrix}
\]\pause
The second to the best - diagonal matrix:
\[
C = \begin{bmatrix}
1&0&0\\
0&2&0\\
0&0&3
\end{bmatrix}
\]
\end{frame}
\begin{frame}
\frametitle{The Hierarchy of Matrices}
Prominent properties of diagonal matrices:\pause
\[
D = \begin{bmatrix}
1&0&0\\
0&2&0\\
0&0&3
\end{bmatrix}^{-1} = \begin{bmatrix}
1&0&0\\
0&1/2&0\\
0&0&1/3
\end{bmatrix}
\]\pause
More generally, 
\[
\begin{bmatrix}
1&0&0\\
0&2&0\\
0&0&3
\end{bmatrix}^k = \begin{bmatrix}
1^k&0&0\\
0&2^k&0\\
0&0&3^k
\end{bmatrix}
\]
Consequently, 
\[
e^D = \begin{bmatrix}
e^1&0&0\\
0&e^2&0\\
0&0&e^3
\end{bmatrix}
\]
\end{frame}

\begin{frame}
\frametitle{The Hierarchy of Matrices}
The almost as good - diagonalizable matrix: 
\[
A = \begin{bmatrix*}[r]
6&-1\\
2&3
\end{bmatrix*} =  \begin{bmatrix*}[r]
2&-1\\
-1&1
\end{bmatrix*}^{-1} \begin{bmatrix*}[r]
5&0\\
0&4
\end{bmatrix*}\begin{bmatrix*}[r]
2&-1\\
-1&1
\end{bmatrix*}
\]
\pause
Properties:
\[
A^k =  \begin{bmatrix*}[r]
2&-1\\
-1&1
\end{bmatrix*}^{-1} \begin{bmatrix*}[r]
5^k&0\\
0&4^k
\end{bmatrix*}\begin{bmatrix*}[r]
2&-1\\
-1&1
\end{bmatrix*}
\]
\pause
\[
e^A =   \begin{bmatrix*}[r]
2&-1\\
-1&1
\end{bmatrix*}^{-1} \begin{bmatrix*}[r]
e^5&0\\
0&e^4
\end{bmatrix*}\begin{bmatrix*}[r]
2&-1\\
-1&1
\end{bmatrix*}
\]
\end{frame}

\begin{frame}
\frametitle{The Hierarchy of Matrices}
A stronger version - orthogonally diagonalizable:
\[
S = \begin{bmatrix*}[r]
1&-2\\[1em]
-2&1
\end{bmatrix*} = \begin{bmatrix*}[r]
\frac{1}{\sqrt 2}&-\frac{1}{\sqrt 2}\\[1em]
\frac{1}{\sqrt 2}&\frac{1}{\sqrt 2}
\end{bmatrix*}^T\begin{bmatrix*}[r]
3&0\\[1em]
0&-1
\end{bmatrix*}\begin{bmatrix*}[r]
\frac{1}{\sqrt 2}&-\frac{1}{\sqrt 2}\\[1em]
\frac{1}{\sqrt 2}&\frac{1}{\sqrt 2}
\end{bmatrix*}
\]
\end{frame}
\end{document}