\documentclass[10pt]{beamer}
\usetheme{PaloAlto}
\usecolortheme{seahorse}
\setbeamertemplate{navigation symbols}{}
\setbeamertemplate{caption}[numbered]
%general package
%\usepackage[utf8]{inputenc}
\usepackage[english]{babel}
\usepackage{geometry}
\usepackage{tcolorbox}
\usepackage[export]{adjustbox}
\usepackage{graphicx}
\graphicspath{{../img}}
\usepackage{graphbox}
%math package
\usepackage{amsmath}
\usepackage{amsfonts}
\usepackage{mathtools}
%\usepackage{amssymb}
%\usepackage{amsthm}
%\usepackage{slashed}
%\usepackage{tikz-cd}
%\usepackage{extarrows}
%font package
\usepackage{mathrsfs}
%\usepackage{bm}
%\usepackage{thmtools}
%misc. package
\usepackage{enumitem}
{
\author[B.H.]{\vspace{1em}\\{\Large MAT215 Intro to Linear Algebra}\\
\vspace{2em}
Instructor: Ben Huang}

\date{}
\title[]{Eigenvalues and Eigenvectors}
\institute[DCC]{\vskip -12em\includegraphics[width = 0.8\textwidth]{DCC.png}}
\logo{\includegraphics[width=0.12\textwidth]{DCC.png}}
%general package
\usepackage[utf8]{inputenc}
\usepackage[english]{babel}
\usepackage{geometry}

%math package
\usepackage{amsmath}
\usepackage{amsfonts}
\usepackage{amssymb}
\usepackage{amsthm}
\usepackage{slashed}
\usepackage{tikz-cd}

%font package
\usepackage{mathrsfs}
\usepackage{bm}

%misc. package
\usepackage{enumitem}
\usepackage{tcolorbox}
\usepackage{etoolbox}
\usepackage{hyperref}
\hypersetup{
  colorlinks=true, urlcolor=blue
}
%declared operators
\DeclareMathOperator{\id}{Id}%identity
\DeclareMathOperator{\ind}{Ind\!}%index
\DeclareMathOperator{\tr}{Tr}%trace
\DeclareMathOperator{\e}{e}%exponential
\DeclareMathOperator{\im}{Im\!}%image
\DeclareMathOperator{\vol}{vol}%volume
\DeclareMathOperator{\cll}{\C\ell}%complexified Clifford algebra
\DeclareMathOperator{\gd}{\slashed{\partial}}%geometric Dirac
\DeclareMathOperator{\D}{\mathcal{D}}%generalized Dirac
\DeclareMathOperator{\Div}{div}%divergence
\DeclareMathOperator{\ud}{\,\mathrm{d}\!}

\DeclareMathOperator{\Hom}{Hom}
\DeclareMathOperator{\xd}{\,d\!}
\DeclareMathOperator{\curl}{curl}
\DeclareMathOperator{\dive}{div}
\DeclareMathOperator{\proj}{proj}


\newcommand{\norm}[1]{\lVert#1\rVert}
\newcommand{\R}{\mathbb R}
\newcommand{\vF}{\mathbf F}
\newcommand{\vv}{\mathbf v}
\newcommand{\inpr}[1]{\left\langle#1\right\rangle}
\newcommand{\fix}{(a,b)}
\newcommand{\uv}{\mathbf u}
\newcommand{\abs}[1]{\lvert #1\rvert}
%texting in citation
\makeatletter
\let\cite\relax
\DeclareRobustCommand{\cite}{%
  \let\new@cite@pre\@gobble
  \@ifnextchar[\new@cite{\@citex[]}}
\def\new@cite[#1]{\@ifnextchar[{\new@citea{#1}}{\@citex[#1]}}
\def\new@citea#1{\def\new@cite@pre{#1}\@citex}
\def\@cite#1#2{[{\new@cite@pre\space#1\if\relax\detokenize{#2}\relax\else, #2\fi}]}
\makeatother

\begin{document}

\frame{\titlepage}

\section{Hierarchy}
\begin{frame}
\frametitle{The Hierarchy of Matrices}
The simplest - scalar matrix:
\[
\begin{bmatrix}
c&0&0\\
0&c&0\\
0&0&c
\end{bmatrix}
\]\pause
The second to the best - diagonal matrix:
\[
C = \begin{bmatrix}
1&0&0\\
0&2&0\\
0&0&3
\end{bmatrix}
\]
\end{frame}
\begin{frame}
\frametitle{The Hierarchy of Matrices}
Prominent properties of diagonal matrices:\pause
\[
D = \begin{bmatrix}
1&0&0\\
0&2&0\\
0&0&3
\end{bmatrix}^{-1} = \begin{bmatrix}
1&0&0\\
0&1/2&0\\
0&0&1/3
\end{bmatrix}
\]\pause
More generally, 
\[
\begin{bmatrix}
1&0&0\\
0&2&0\\
0&0&3
\end{bmatrix}^k = \begin{bmatrix}
1^k&0&0\\
0&2^k&0\\
0&0&3^k
\end{bmatrix}
\]
Consequently, 
\[
e^D = \begin{bmatrix}
e^1&0&0\\
0&e^2&0\\
0&0&e^3
\end{bmatrix}
\]
\end{frame}

\begin{frame}
\frametitle{The Hierarchy of Matrices}
The almost as good - diagonalizable matrix: 
\[
A = \begin{bmatrix*}[r]
6&-1\\
2&3
\end{bmatrix*} =  \begin{bmatrix*}[r]
1&1\\
1&2
\end{bmatrix*}
\begin{bmatrix*}[r]
5&0\\
0&4
\end{bmatrix*}
\begin{bmatrix*}[r]
1&1\\
1&2
\end{bmatrix*}^{-1}
\]
\pause
Properties:
\[
A^k =  \begin{bmatrix*}[r]
1&1\\
1&2
\end{bmatrix*}
\begin{bmatrix*}[r]
5^k&0\\
0&4^k
\end{bmatrix*}
\begin{bmatrix*}[r]
1&1\\
1&2
\end{bmatrix*}^{-1}
\]
\pause
\[
e^A =  \sum_{k = 0}^\infty \frac{1}{k!}A^k= \begin{bmatrix*}[r]
1&1\\
1&2
\end{bmatrix*}
\begin{bmatrix*}[r]
e^5&0\\
0&e^4
\end{bmatrix*}
\begin{bmatrix*}[r]
1&1\\
1&2
\end{bmatrix*}^{-1}
\]
\end{frame}

\begin{frame}
\frametitle{The Hierarchy of Matrices}
A stronger version - orthogonally diagonalizable:
\[
S = \begin{bmatrix*}[r]
1&-2\\[1em]
-2&1
\end{bmatrix*} = \begin{bmatrix*}[r]
\frac{1}{\sqrt 2}&\frac{1}{\sqrt 2}\\[1em]
-\frac{1}{\sqrt 2}&\frac{1}{\sqrt 2}
\end{bmatrix*}\begin{bmatrix*}[r]
3&0\\[1em]
0&-1
\end{bmatrix*}\begin{bmatrix*}[r]
\frac{1}{\sqrt 2}&\frac{1}{\sqrt 2}\\[1em]
-\frac{1}{\sqrt 2}&\frac{1}{\sqrt 2}
\end{bmatrix*}^T
\]
\pause
Applications:
\begin{enumerate}[label = {(\alph*)}]
\item Classify quadric curves and surfaces\pause
\item Simplify the inertia tensor of a rigid body (classical mechanics)\pause
\item Principal Component Analysis
\end{enumerate}
\end{frame}
\begin{frame}
\frametitle{The Hierarchy of Matrices}
Advanced decompositions:
\begin{enumerate}[label=\bullet]
\item
Singular Value decomposition
{\footnotesize
\[
\begin{bmatrix*}[r]
3&2&2\\
2&3&-2
\end{bmatrix*} = \begin{bmatrix*}[r]
1/\sqrt 2&1/\sqrt 2\\
1/\sqrt 2 & -1/\sqrt 2
\end{bmatrix*} \begin{bmatrix*}[r]
5&0&0\\
0&3&0
\end{bmatrix*}\begin{bmatrix*}[r]
1/\sqrt 2 & 1/\sqrt{18}&2/3\\
1/\sqrt 2 & -1/\sqrt{18} &-2/3\\
0&4/\sqrt{18}&-1/3
\end{bmatrix*}^T
\]
}
\item
Jordan canonical form
\[
\begin{bmatrix*}[r]
-2&2&1\\
-7&4&2\\
5&0&0
\end{bmatrix*} = \begin{bmatrix*}[r]
0&1&1\\
-1&-1&2\\
2&5&0
\end{bmatrix*}
\begin{bmatrix*}[r]
0&0&0\\
0&1&1\\
0&0&1
\end{bmatrix*}
\begin{bmatrix*}[r]
0&1&1\\
-1&-1&2\\
2&5&0
\end{bmatrix*}^{-1}
\]
\end{enumerate}
\end{frame}

\section{Eigenvalues}
\begin{frame}
\frametitle{Meet the Eigenvalues}
{\bf Question:} How to even start diagonalizing a matrix?\pause

{\bf The Trick:} Reverse Engineering! \pause

Suppose $A =P\begin{bmatrix}
\lambda_1&0\\
0&\lambda_2
\end{bmatrix}P^{-1}$, 
where $P =  \begin{bmatrix}
p_{11}&p_{12}\\
p_{21}&p_{22}
\end{bmatrix} = \begin{bmatrix}
\mathbf p_1&\mathbf p_2
\end{bmatrix}$.\pause
\begin{align*}
\onslide<4->{
&A\mathbf p_1 = P\begin{bmatrix}
\lambda_1&0\\
0&\lambda_2
\end{bmatrix}P^{-1}\mathbf p_1
}\\
\onslide<5->{
&A\mathbf p_1 = P\begin{bmatrix}
\lambda_1&0\\
0&\lambda_2
\end{bmatrix}\begin{bmatrix}
1\\0
\end{bmatrix}
}\\
\onslide<6->{
&A\mathbf p_1 = P\begin{bmatrix}
\lambda_1\\
0
\end{bmatrix} =  \begin{bmatrix}
p_{11}&p_{12}\\
p_{21}&p_{22}
\end{bmatrix}\begin{bmatrix}
\lambda_1\\
0
\end{bmatrix}
}\\
\onslide<7->{
&A\mathbf p_1 = \begin{bmatrix}
\lambda_1p_{11}\\
\lambda_1p_{21}
\end{bmatrix} = \lambda_1\mathbf p_1
}
\end{align*}
\end{frame}
\begin{frame}
\frametitle{Meet the Eigenvalues}
\begin{tcolorbox}
{\bf Definition} Let $A$ be a square matrix. $\lambda$ is called an {\bf eigenvalue} of $A$ if there is a non-zero column vector $\mathbf v$ such that 
\[
A\mathbf v = \lambda\mathbf v,
\]
and $\mathbf v$ is called an {\bf eigenvector} of $A$ associated with $\lambda$.
\end{tcolorbox}\pause

\vspace{1em}
\href{https://webwork.messiah.edu/webwork2/MATH261_2025SP/Eigenvalues_and_Eigenvectors/1}{Exercises on WeBWork}

\vspace{1em}
\pause
\href{https://www.geogebra.org/3d/q9rzs2af}{A 3-D Example on GeoGebra}
\end{frame}

\begin{frame}
\frametitle{Meet the Eigenvalues}
How to find the eigenvalues?\pause
\[
\lambda\mathbf v = A\mathbf v\pause \Leftrightarrow \lambda\mathbf v - A\mathbf v = \mathbf 0\pause \Leftrightarrow
  \lambda I\mathbf v - A\mathbf v  = \mathbf 0 \pause \Leftrightarrow ( \lambda I - A)\mathbf v = \mathbf 0 \]
\pause
\[
\mathbf v \ne \mathbf 0\pause  \Leftrightarrow \lambda I - A\text{ is nonsingular}\pause \Leftrightarrow \det(\lambda I - A) = 0.
\]
\pause
\begin{tcolorbox}
{\bf Definition} The polynomial $p(\lambda) = \det(\lambda I - A)$ is called the {\bf characteristic polynomial} of $A$. Note that the roots of $p(\lambda)$ are precisely the eigenvalues of $A$.
\end{tcolorbox}
\end{frame}

\section{Diagonalization}
\begin{frame}
\frametitle{Diagonalize a Matrix}

{\bf Example.} 

Let $A = \begin{bmatrix*}[r]
6&-3\\
-2&1
\end{bmatrix*}$. Diagonalize $A$.\pause

{\bf Solution:}

Step 1: Find the eigenvalues of $A$.
\[
\det(\lambda I - A) = \det\left(\begin{bmatrix*}[r]
\lambda - 6&3\\
2&\lambda - 1
\end{bmatrix*}\right) = \pause \lambda^2 - 7\lambda = 0;
\]
$\lambda_1 = 0$, $\lambda_2 = 7$.\pause

Step 2: Find a basis for each eigenspace.\pause

$\lambda_1 = 0$:\pause
\[
 (0 I - A)\mathbf v = \mathbf 0
\]
\pause
\[
\begin{bmatrix*}[r]
-6&3\\
2&-1
\end{bmatrix*}\begin{bmatrix}
x_1\\
x_2
\end{bmatrix} = \begin{bmatrix}
0\\0	
\end{bmatrix}
\]\pause
\[
\begin{bmatrix}
x_1\\
x_2	
\end{bmatrix} = \begin{bmatrix}
\frac{1}{2}t\\
t
\end{bmatrix} = t \begin{bmatrix}
\frac{1}{2}\\
1
\end{bmatrix}, \text{ thus } \mathscr B_1 = \left\{ \begin{bmatrix}
\frac{1}{2}\\
1
\end{bmatrix}\right\}.
\]

\end{frame}

\begin{frame}
\frametitle{Diagonalize a Matrix}

$\lambda_2 = 7$:\pause
\[
 (7 I - A)\mathbf v = \mathbf 0
\]
\pause
\[
\begin{bmatrix*}[r]
1&3\\
2&6
\end{bmatrix*}\begin{bmatrix}
x_1\\
x_2
\end{bmatrix} = \begin{bmatrix}
0\\0	
\end{bmatrix}
\]\pause
\[
\begin{bmatrix}
x_1\\
x_2	
\end{bmatrix} = \begin{bmatrix}
-3t\\
t
\end{bmatrix} = t \begin{bmatrix}
-3\\
1
\end{bmatrix}, \text{ thus } \mathscr B_2 = \left\{ \begin{bmatrix}
-3\\
1
\end{bmatrix}\right\}.
\]\pause

Step 4: Since the number of basis vectors is the same as the dimension of the ambient space ($R^2$), $A$ is diagonalizable, and 
\[
D = P^{-1}AP, 
\] where $D = \begin{bmatrix}
0&0\\
0&7	
\end{bmatrix}$, $P = \begin{bmatrix}
\frac{1}{2}&-3\\
1&1
\end{bmatrix}$.
\end{frame}
\end{document}