\documentclass[10pt]{article}

\usepackage{geometry}
\geometry{margin=3em}
\usepackage[export]{adjustbox}
\usepackage{array}
\usepackage{amsmath}
\usepackage{amsfonts}
\usepackage{fancyhdr}
\pagestyle{empty}
\usepackage{lastpage}
\usepackage{xcolor}
\usepackage{enumitem}
\usepackage{pifont}
\usepackage{graphicx}
\graphicspath{{../img}}
\usepackage{pgfplots}
\pgfplotsset{compat=1.18} 
\usepackage{tabularx}
\usepackage{tikz}
\usetikzlibrary{patterns}
\usepackage{luacode}
\newcommand{\studentname}{}
\newcommand{\classtitle}{}
\newcommand{\csvsourceOne}{comprehensive-roster.csv}

\begin{luacode}
local ftcsv = require("ftcsv")
behaviorRecords = ftcsv.parse("\csvsourceOne", {headers=false})
headers = behaviorRecords[1]
\end{luacode}

\begin{document}
\begin{luacode*}
tex.print("\\footnotesize\\renewcommand{\\arraystretch}{2}\\begin{tabularx}{.6\\textwidth}{|>{\\centering\\arraybackslash}X|>{\\centering\\arraybackslash}X|>{\\centering\\arraybackslash}X|>{\\centering\\arraybackslash}X|>{\\centering\\arraybackslash}X|>{\\centering\\arraybackslash}X|}")
tex.print("\\hline")
for i = 3, 8 do
	tex.print(headers[i])
	if i ~= 8 then
		tex.print("&")
	else 
	 	tex.print("\\\\ \\hline")
	end
end
tex.print("\\end{tabularx}")
\end{luacode*}

\end{document}