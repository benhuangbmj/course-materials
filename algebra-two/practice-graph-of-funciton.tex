\documentclass[10pt]{article}

\usepackage{geometry}
\geometry{margin=3em,top=6em, left=2.5cm, headheight=\paperheight}
\usepackage[export]{adjustbox}
\usepackage{array}
\usepackage{amsmath}
\usepackage{amsfonts}
\usepackage{fancyhdr}
\pagestyle{fancy}
\fancyhf{}
\lhead{Algebra II}
\chead{Graph of Function}
\rhead{Practice, Page \thepage}
\usepackage{lastpage}
\usepackage{xcolor}
\usepackage{enumitem}
\usepackage{pifont}
\usepackage{graphicx}
\graphicspath{{../img}}
\usepackage{pgfplots}
\pgfplotsset{compat=1.18}
\usepackage{tabularx}
\usepackage{luacode}

\newcommand{\R}{\mathbb R}
\newcommand{\e}{{\rm e}}
\newcommand{\pobr}[1]{\left\langle#1\right\rangle}
\newcommand{\norm}[1]{\lVert #1 \rVert}
\newcommand{\abs}[1]{\lvert #1 \rvert}

\DeclareMathOperator{\xd}{d\!}
\DeclareMathOperator{\proj}{proj}

\title{}
\date{}

\begin{document}
\noindent
{
Name \rule{16em}{.5pt}\hspace{\stretch{1}} Date \rule{8em}{.5pt}\hspace{\stretch{1}} Period \rule{4em}{.5pt}
}
\vspace{1em}

{\noindent\bf Targets.}
\begin{itemize}
    \item to plot the points on the graph of a function
    \item to sketch the graph by tracing the points
\end{itemize}
{\noindent\bf Do Now.}
Use this coordinate frame for the following problems. 
\begin{center}
\begin{tikzpicture}
\begin{axis}[
    xlabel={$x$},
    ylabel={$y$},
    grid=both,
    minor tick num=1,
    axis lines=middle,
    xmin=-6,xmax=6,
    ymin=-6,ymax=6,
    domain=-5:5,
    samples=100,
    width=0.6\textwidth,
    grid style={draw=gray!80},
]
\addplot[
    only marks,
    mark=*,
    mark size=3pt,
    color=black
] coordinates {(1,2)};
\end{axis}
\end{tikzpicture}
\end{center}   
\begin{enumerate}
    \item Find the coordinates of the black point in the coordinate frame.
    \item Plot the point \((-4, -3)\) in the coordinate frame.
\end{enumerate}

{\noindent\bf Take note.}\\[1em]

\noindent
\rule{\textwidth}{.5pt}\\[1em]
\rule{\textwidth}{.5pt}\\[1em]
\rule{\textwidth}{.5pt}\\[1em]
\rule{\textwidth}{.5pt}\\[1em]
\rule{\textwidth}{.5pt}\\[1em]
\rule{\textwidth}{.5pt}\\[1em]
\rule{\textwidth}{.5pt}\\[1em]
\rule{\textwidth}{.5pt}\\[1em]
\rule{\textwidth}{.5pt}\\[1em]
\rule{\textwidth}{.5pt}\\[1em]
\rule{\textwidth}{.5pt}\\[1em]
\rule{\textwidth}{.5pt}
\clearpage

{\noindent\bf Practice.}

Let $f(x) = 2x$.
\begin{enumerate}
    \item Complete the following table.
    \item Plot the points in the coordinate frame.
    \item Sketch the graph of $f(x)$ by tracing the points.
\end{enumerate}
\begin{luacode*}
tex.print("\\begin{center}\\Large")
tex.print("\\begin{tabular}{|c|c|}")
tex.print("\\hline\\textbf{$x$} & \\textbf{$f(x)$} \\\\ \\hline")

for i = -6, 6 do
  local x = i * 0.5
  tex.print(string.format("%g & \\\\ \\hline", x))
end

tex.print("\\end{tabular}")
tex.print("\\end{center}")
\end{luacode*}
\begin{center}
\begin{tikzpicture}
\begin{axis}[
    xlabel={$x$},
    ylabel={$y$},
    grid=both,
    minor tick num=1,
    axis lines=middle,
    xmin=-6,xmax=6,
    ymin=-6,ymax=6,
    domain=-5:5,
    samples=100,
    width=0.6\textwidth,
    grid style={draw=gray!80},
]
\end{axis}
\end{tikzpicture}
\end{center} 
{\noindent\bf Exit Ticket.}
Use the approach we learned in this lesson to sketch $f(x) = (x-1)^2$.

\end{document}