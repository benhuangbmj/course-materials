\documentclass[twoside, 10pt]{article}

\usepackage{geometry}
\geometry{outer=3em, inner=2.2cm, top=6em, bottom=4em, headheight=\paperheight}
\usepackage[export]{adjustbox}
\usepackage{array}
\usepackage{amsmath}
\usepackage{amsfonts}
\usepackage{fancyhdr}
\pagestyle{fancy}
\fancyhf{}
\lhead{Algebra II - BASE}
\chead{Inverse Function}
\rhead{Pretest, Page \thepage}
\usepackage{lastpage}
\usepackage{xcolor}
\usepackage{enumitem}
\usepackage{pifont}
\usepackage{graphicx}
\graphicspath{{../img}}
\usepackage{pgfplots}
\pgfplotsset{compat=1.18}
\usepackage{tabularx}
\usepackage{tikz}
\usetikzlibrary{patterns}

\newcommand{\R}{\mathbb R}
\newcommand{\e}{{\rm e}}
\newcommand{\pobr}[1]{\left\langle#1\right\rangle}
\newcommand{\norm}[1]{\lVert #1 \rVert}
\newcommand{\abs}[1]{\lvert #1 \rvert}

\DeclareMathOperator{\xd}{d\!}
\DeclareMathOperator{\proj}{proj}

\title{}
\date{}

\begin{document}
\noindent
{\large
First Name \rule{6em}{.1pt}\hspace{\stretch{1}}Last Name \rule{6em}{.1pt}\hspace{\stretch{1}} Due \rule{1.5em}{.1pt} -- \rule{1.5em}{.1pt} -- \rule{1.5em}{.1pt}\hspace{\stretch{1}} Period \rule{2em}{.1pt}\hspace{\stretch{1}} Score \rule{2em}{.1pt}
}
\vspace{1em}

\begin{enumerate}
\item Use the quadratic formula $\displaystyle x = \frac{-b\pm\sqrt{b^2 - 4ac}}{2a}$ to solve the equation $2x^2+4x -6 = 0$.
\vspace{\stretch{1}}

\item Determine if the pair of functions are inverses of each other. SHOW WORK and explain how you know.
\begin{enumerate}
\item
$f(x) = 0.5x-4$,\quad $g(x) = 4x +16$.
\vspace{\stretch{1}}
\item 
$f(x) = x^2 - 2x$ on $[1, \infty)$,\quad $g(x) = 1 + \sqrt{x+1}$ on $[-1, \infty)$. 
\vspace{\stretch{1}}
\end{enumerate} 
\item
The signed distance of a train from Station $A$, written as $d(t)$ (in miles), depends on time $t$ (in hours) and is given by 
\[
d(t) = 70t - 200
\]
A negative distance means the train is {\it west} of Station $A$, and a positive distance means it is {\it east} of Station $A$.

\begin{enumerate}
\item Find the inverse of $d(t)$. What does this new function tell you in words?
\vspace{\stretch{1}}

\item Station $B$ is on the same track, $10$ miles {\bf east} of Station $A$. Use the inverse function to figure out when the train reaches Station $B$.
\vspace{\stretch{1}}

\item Station $C$ is on the same track, $130$ miles {\bf west} of Station $A$. Use the inverse function to figure out when the train reaches Station $C$.
\end{enumerate}
\vspace{\stretch{1}}
\end{enumerate}

\end{document}