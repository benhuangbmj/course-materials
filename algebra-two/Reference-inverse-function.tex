\documentclass[twoside, 10pt]{article}

\usepackage{geometry}
\geometry{outer=3em, inner=2.2cm, top=6em, bottom=4em, headheight=\paperheight}
\usepackage[export]{adjustbox}
\usepackage{array}
\usepackage{amsmath}
\usepackage{amsfonts}
\usepackage{fancyhdr}
\pagestyle{fancy}
\fancyhf{}
\lhead{Algebra II - BASE}
\chead{Inverse Function}
\rhead{Reference, Page \thepage}
\usepackage{lastpage}
\usepackage{xcolor}
\usepackage{enumitem}
\usepackage{pifont}
\usepackage{graphicx}
\graphicspath{{../img}}
\usepackage{pgfplots}
\pgfplotsset{compat=1.18}
\usepackage{tabularx}
\usepackage{tikz}
\usetikzlibrary{patterns}

\newcommand{\R}{\mathbb R}
\newcommand{\e}{{\rm e}}
\newcommand{\pobr}[1]{\left\langle#1\right\rangle}
\newcommand{\norm}[1]{\lVert #1 \rVert}
\newcommand{\abs}[1]{\lvert #1 \rvert}

\DeclareMathOperator{\xd}{d\!}
\DeclareMathOperator{\proj}{proj}

\title{}
\date{}

\begin{document}
\begin{enumerate}
\item
{\bf Verify inverse functions.} A pair of inverse functions undo each other. In diagram, it means
{
\setlength{\fboxsep}{1em}
\[ x\quad\substack{\xrightarrow{\hspace{5em}}\\ \xleftarrow{\hspace{5em}}} \quad\fbox{f(x)}\quad\substack{\xrightarrow{\hspace{5em}}\\ \xleftarrow{\hspace{5em}}} \quad\fbox{g(x)}\quad\substack{\xrightarrow{\hspace{5em}}\\ \xleftarrow{\hspace{5em}}} \quad x\]
}
For example,
\begin{center}
\setlength{\fboxsep}{1em}
$x$ $\xrightarrow{\hspace{3em}}$ $\overbrace{\fbox{$x+1$}}^{f(x)}$ $\xrightarrow{\parbox{4em}{\centering\tiny$(x) + 1$}}$ $x+1$ $\xrightarrow{\hspace{3em}}$ $\underbrace{\fbox{$x-1$}}_{g(x)}$ $\xrightarrow{\parbox{4em}{\centering\tiny$(x+1) - 1$}}$ $x$

\setlength{\fboxsep}{1em}
$x$ $\xrightarrow{\hspace{3em}}$ $\overbrace{\fbox{$x-1$}}^{g(x)}$ $\xrightarrow{\parbox{4em}{\centering\tiny$(x)- 1$}}$ $x-1$ $\xrightarrow{\hspace{3em}}$ $\underbrace{\fbox{$x+1$}}_{f(x)}$ $\xrightarrow{\parbox{4em}{\centering\tiny$(x-1) + 1$}}$ $x$
\end{center}
Thus, $f(x) = x+1$ and $g(x) = x-1$ are the inverse functions of each other.
\item 
{\bf Find the inverse function.} The gist of finding the inverse function is in {\bf expressing the independent variable $\mathbf x$ in terms of the dependent variable $\mathbf y$.} For example, let $y = f(x) = 2x - 5$, we can express $x$ by $y$ through the following algebra.
\begin{align*}
y + 5 = 2x - 5 + 5 &\Longrightarrow y + 5 = 2x\\
\left(\frac{1}{2}\right) (y + 5) = \left(\frac{1}{2}\right) 2x &\Longrightarrow \frac{y+5}{2} = x
\end{align*}
Lastly, we swap $x$ and $y$ to maintain the consistency in the choice of symbols. Thus, we conclude that $f^{-1}(x) = (x+5)/2$. For another example, let $y= g(x) = x^2 + 5$ on $x$ in $[0, \infty)$,
\begin{align*}
y - 5 = x^2 + 5 - 5 &\Longrightarrow y-5 = x^2 \\
\sqrt{y - 5} = \sqrt{x^2} &\Longrightarrow \sqrt{y - 5} = x
\end{align*}
Thus, $g^{-1}(x) = \sqrt{x-5}$ on $x$ in $[5, \infty)$.

\textbf{Final Remark.} The symbol used for the input of a function is a \textit{dummy variable}. This means the choice of symbol does not affect the function itself, as long as we use the symbol consistently within an expression. In other words, it serves only as a placeholder. For example, the function written as
\[
f^{-1}(x) = \frac{x+5}{2}
\]
is completely equivalent to any of the following:
\[
f^{-1}(y) = \frac{y+5}{2}, \qquad
f^{-1}(t) = \frac{t+5}{2}, \qquad
f^{-1}(\star) = \frac{\star+5}{2}.
\]

\end{enumerate}
\end{document}