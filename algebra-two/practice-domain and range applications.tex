\documentclass[twoside, 10pt]{article}

\usepackage{geometry}
\geometry{outer=3em, inner=2.2cm, top=6em, bottom=4em, headheight=\paperheight}
\usepackage[export]{adjustbox}
\usepackage{array}
\usepackage{amsmath}
\usepackage{amsfonts}
\usepackage{fancyhdr}
\pagestyle{fancy}
\fancyhf{}
\lhead{Algebra II - BASE}
\chead{Domain and Range Applications}
\rhead{Practice, Page \thepage}
\usepackage{lastpage}
\usepackage{xcolor}
\usepackage{enumitem}
\usepackage{pifont}
\usepackage{graphicx}
\graphicspath{{../img}}
\usepackage{pgfplots}
\pgfplotsset{compat=1.18}
\usepackage{tabularx}
\usepackage{tikz}
\usetikzlibrary{patterns}

\newcommand{\R}{\mathbb R}
\newcommand{\e}{{\rm e}}
\newcommand{\pobr}[1]{\left\langle#1\right\rangle}
\newcommand{\norm}[1]{\lVert #1 \rVert}
\newcommand{\abs}[1]{\lvert #1 \rvert}

\DeclareMathOperator{\xd}{d\!}
\DeclareMathOperator{\proj}{proj}

\title{}
\date{}

\begin{document}
\noindent
{\large
First Name \rule{6em}{.1pt}\hspace{\stretch{1}}Last Name \rule{6em}{.1pt}\hspace{\stretch{1}} Date \rule{1.5em}{.1pt} -- \rule{1.5em}{.1pt} -- \rule{1.5em}{.1pt}\hspace{\stretch{1}} Period \rule{2em}{.1pt}\hspace{\stretch{1}} Score \rule{2em}{.1pt}
}
\vspace{1em}

\noindent
{\bf Learning Objectives.}
\begin{itemize}
\item
Find the domain and range of a function arising from real-world applications and set the graphing window accordingly.
\item
Discover the relationship between the $x$-intercepts and the vertex of a parabola.
\end{itemize}

\noindent
{\bf Do Now.} {\it Work on the following problem while the teacher is taking attendance and returning work.}

Dr. Benjamin wants to add a new feature to his video game in which the player can throw a stone toward the enemy. According to your daily experience and common sense, draw the trajectory of the stone as it flies through the air.
\vspace{\stretch{3}}

\noindent
{\bf Discussion.}
According to physics, a stone flying through the air follows the path of part of a {\it parabola}, which is the graph of a quadratic function $y=ax^2 + bx + c$, where $a<0$. The following figure shows the graph of a typical parabola. Examine the graph and determine the relationships among the $x$-intercepts, the axis of symmetry, and the maximum point.
\vspace{2em}

\begin{tikzpicture}
\begin{axis}[
axis lines=middle,
xlabel={$x$},
ylabel={$y$},
ticks=none,
ymax=30,
ymin=-20,
domain=-2.5:1.5,
]
\addplot[thick]{-10*(x+2)*(x-1)};
\draw[dashed](-0.5, -25) -- (-0.5, 30);
\draw (-2,0) circle(8pt);
\draw (1,0) circle(8pt);
\draw[fill=black] (-2,0) circle(2pt);
\draw[fill=black] (1,0) circle(2pt);
\end{axis}
\end{tikzpicture}
\vspace{1em}

{\bf Your observations:}

\vspace{\stretch{1}}

\clearpage

\noindent
{\bf Practice.}
In Algebra I, we learned the quadratic formula. The solutions to the quadratic equation $\displaystyle ax^2 + bx + c = 0$ are given by
\[
x = \frac{-b \pm \sqrt{b^2 - 4ac}}{2a}
\]
assuming $b^2 - 4ac\geq0$.
\begin{enumerate}
\item
How would you adapt the quadratic formula to solve $ax^2 + bx + c = d$, where $d\ne0$?
\vspace{\stretch{1}}
\item Let the ground be on the $x$-axis. Suppose a stone thrown from a height of 6 units follows part of the parabola $\displaystyle y = f(x) = -\frac{x^2}{6} + \frac{8x}{3} + 6$. Determine the realistic domain and range of the function $f(x)$.

{\it Hint: First find the $x$-intercepts. Then use the relationship you discovered in the Discussion section to find the vertex.}
\vspace{\stretch{1}}
\item Use the domain and range you found in the previous problem to set the window on your graphing calculator appropriately, then graph the function $f(x)$.
\vspace{\stretch{1}}
\end{enumerate}
\end{document}