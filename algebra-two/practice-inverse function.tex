\documentclass[twoside, 10pt]{article}

\usepackage{geometry}
\geometry{outer=3em, inner=2.2cm, top=6em, bottom=4em, headheight=\paperheight}
\usepackage[export]{adjustbox}
\usepackage{array}
\usepackage{amsmath}
\usepackage{amsfonts}
\usepackage{fancyhdr}
\pagestyle{fancy}
\fancyhf{}
\lhead{Algebra II - BASE}
\chead{Inverse Function, I}
\rhead{Practice, Page \thepage}
\usepackage{lastpage}
\usepackage{xcolor}
\usepackage{enumitem}
\usepackage{pifont}
\usepackage{graphicx}
\graphicspath{{../img}}
\usepackage{pgfplots}
\pgfplotsset{compat=1.18}
\usepackage{tabularx}
\usepackage{tikz}
\usetikzlibrary{patterns}

\newcommand{\R}{\mathbb R}
\newcommand{\e}{{\rm e}}
\newcommand{\pobr}[1]{\left\langle#1\right\rangle}
\newcommand{\norm}[1]{\lVert #1 \rVert}
\newcommand{\abs}[1]{\lvert #1 \rvert}

\DeclareMathOperator{\xd}{d\!}
\DeclareMathOperator{\proj}{proj}

\title{}
\date{}

\begin{document}
\noindent
{\large
First Name \rule{6em}{.1pt}\hspace{\stretch{1}}Last Name \rule{6em}{.1pt}\hspace{\stretch{1}} Date \rule{1.5em}{.1pt} -- \rule{1.5em}{.1pt} -- \rule{1.5em}{.1pt}\hspace{\stretch{1}} Period \rule{2em}{.1pt}\hspace{\stretch{1}} Score \rule{2em}{.1pt}
}
\vspace{1em}

\noindent
{\bf Skill Builder.} {\it Work on the Skill Builder problems on the screen while the teacher is taking attendance and returning work.}\\[.25em]

\noindent
\rule{\textwidth}{.1pt}\\[1em]
\rule{\textwidth}{.1pt}\\[1em]
\rule{\textwidth}{.1pt}\\[1em]
\rule{\textwidth}{.1pt}\\[1em]
\rule{\textwidth}{.1pt}\\[1em]
\rule{\textwidth}{.1pt}\\[1em]

\noindent
{\bf Learning Objectives.}
\begin{itemize}
\item
Verify a pair of inverse functions via composition
\end{itemize}

\noindent
{\bf Discussion.}

Find the operation that undoes the given operation.
\begin{enumerate}
\item
Adding 2.
\vspace{\stretch{1}}
\item
Multiplying by 10.
\vspace{\stretch{1}}
\item
Squaring a negative number.
\vspace{\stretch{1}}
\end{enumerate}
\noindent
{\bf Concepts.}
The {\it inverse function} undoes the original function. Formally, $g(x)$ is called the {\bf inverse function} of $f(x)$ if both of the following are true:
\begin{enumerate}
\item
$(f\circ g)(x) = x$
\item
$(g\circ f)(x) = x$
\end{enumerate}
See the diagram below.
{\Large
\setlength{\fboxsep}{1em}
\[ x\quad\substack{\xrightarrow{\hspace{5em}}\\ \xleftarrow{\hspace{5em}}} \quad\fbox{\hspace{.5em} f(x) \hspace{.5em}}\quad\substack{\xrightarrow{\hspace{5em}}\\ \xleftarrow{\hspace{5em}}} \quad\fbox{\hspace{.5em} g(x) \hspace{.5em}}\quad\substack{\xrightarrow{\hspace{5em}}\\ \xleftarrow{\hspace{5em}}} \quad x\]
}
The inverse function of $f(x)$ is denoted by $f^{-1}(x)$. It is worthwhile to note that
\begin{itemize}
\item
domain($f(x)$) = range($f^{-1}(x)$)
\item
domain($f^{-1}(x)$) = range($f(x)$)
\end{itemize}
Moreover, the graph of $f(x)$ and the graph of $f^{-1}(x)$ are symmetric about the line $y=x$.
\clearpage

\noindent
{\bf Examples.}

Verify the pair of functions below are the inverse functions of each other.
\begin{enumerate}
\item
$f(x) = x + 2$, $g(x) = x - 2$.
\vspace{\stretch{1}}
\item $f(x) = x\times 10$, $g(x) = x\div 10$.
\vspace{\stretch{1}}
\end{enumerate}

\noindent
{\bf Challenge.} Is $g(x) = \sqrt{x}$ the inverse function of $f(x) = x^2$? Why or why not?
\vspace{\stretch{1}}

\end{document}