\documentclass[twoside, 10pt]{article}

\usepackage{geometry}
\geometry{outer=3em, inner=2.2cm, top=6em, bottom=4em, headheight=\paperheight}
\usepackage[export]{adjustbox}
\usepackage{array}
\usepackage{amsmath}
\usepackage{amsfonts}
\usepackage{fancyhdr}
\pagestyle{fancy}
\fancyhf{}
\lhead{Algebra II - BASE}
\chead{Function  Characteristics - Business Applications}
\rhead{Pretest, Page \thepage}
\usepackage{lastpage}
\usepackage{xcolor}
\usepackage{enumitem}
\usepackage{pifont}
\usepackage{graphicx}
\graphicspath{{../img}}
\usepackage{pgfplots}
\pgfplotsset{compat=1.18}
\usepackage{tabularx}
\usepackage{tikz}
\usetikzlibrary{patterns}
\usepackage{luacode}

\newcommand{\R}{\mathbb R}
\newcommand{\e}{{\rm e}}
\newcommand{\pobr}[1]{\left\langle#1\right\rangle}
\newcommand{\norm}[1]{\lVert #1 \rVert}
\newcommand{\abs}[1]{\lvert #1 \rvert}

\DeclareMathOperator{\xd}{d\!}
\DeclareMathOperator{\proj}{proj}

\title{}
\date{}

\begin{document}
\noindent
{\large
First Name \rule{6em}{.1pt}\hspace{\stretch{1}}Last Name \rule{6em}{.1pt}\hspace{\stretch{1}} Date \rule{1.5em}{.1pt} -- \rule{1.5em}{.1pt} -- \rule{1.5em}{.1pt}\hspace{\stretch{1}} Period \rule{2em}{.1pt}\hspace{\stretch{1}} Score \rule{2em}{.1pt}
}
\vspace{1em}

{\noindent\bf Thoughts of the Day}
\begin{center}
{\it Exams are not lotteries.}
\end{center}

{\noindent \bf Learning Objectives.}
\begin{itemize}
\item
Use the graph of a business function to make informed business decisions.
\end{itemize}
{\noindent\bf Do Now.}
Let $f(x) = x^2 - x$.
\begin{enumerate}
    \item Complete the following table.
    \item Plot the points in the coordinate frame.
    \item Sketch the graph of $f(x)$ on $[-2, 2]$ by tracing the points.
\end{enumerate}
\begin{luacode*}
tex.print("\\begin{center}\\begingroup\\renewcommand{\\arraystretch}{2}")
tex.print("\\begin{tabular}{cc}")
tex.print("\\begin{tabular}{|c|c|}")
tex.print("\\hline\\textbf{$x$} & \\textbf{$f(x)$} \\\\ \\hline")

for i = -4, 4 do
  local x = i * 0.5
  tex.print(string.format("%g & \\hspace{5em} \\\\ \\hline", x))
end
tex.print("\\end{tabular}&\\hspace{5em}")
tex.print("\\begin{tikzpicture}[baseline={(current bounding box.center)}]")
tex.print("\\begin{axis}[    xlabel={$x$},    ylabel={$y$},    grid=both,    minor tick num=1,    axis lines=middle,     xmin=-6,xmax=6,    ymin=-6,ymax=6,    domain=-2:2,    samples=100,    width=0.6\\textwidth,    grid style={draw=gray!80, dashed},]\\end{axis}\\end{tikzpicture}")
tex.print("\\end{tabular}\\endgroup")
tex.print("\\end{center}")
\end{luacode*}

{\noindent\bf Problems.}
A manufacturer of sweatshirts finds that profits and costs fluctuate depending on the number
of products created. Creating more products doesn't always increase profits because it requires
additional costs, such as building a larger facility or hiring more workers. The manufacturer
determines the profit, $p(x)$, in thousands of dollars, as a function of the number of sweatshirts
sold, $x$, in thousands. This function, $p$, is given below.
\[
p(x) = -x^3+11x^2-7x-69
\]
\begin{enumerate}
\item
$(1, -66)$ is a point on the graph of the function. Interpret the meaning of this set of coordinates in the business language. 
\vspace{\stretch{1}}
\clearpage
\item
Graph $y=p(x)$, over the interval $0\leq x \leq 9$, on the set of axes below.
\begin{center}
\begin{tikzpicture}
\begin{axis}[
axis lines=middle,
xmin=-7, xmax = 13,
xlabel={$x$}, ylabel={$y$},
ymin=-90, ymax=180,
grid=both,
ytick distance=30,
minor tick num=1,
width=0.8\textwidth
]
\end{axis}
\end{tikzpicture}
\end{center}
\item Over the given interval, state the coordinates of the maximum of p and round all values to
the nearest integer. Explain what this point represents in terms of the number of sweatshirts sold
and profit.
\vspace{\stretch{1}}
\item Determine how many sweatshirts, to the nearest whole sweatshirt, the manufacturer would need
to produce in order to first make a positive profit. Justify your answer.
\vspace{\stretch{1}}
\end{enumerate}

\end{document}