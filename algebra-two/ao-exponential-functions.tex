\documentclass[twoside, 10pt]{article}

\usepackage{geometry}
\geometry{outer=3em, inner=2.2cm, top=6em, bottom=4em, headheight=\paperheight}
\usepackage[export]{adjustbox}
\usepackage{array}
\usepackage{amsmath}
\usepackage{amsfonts}
\usepackage{fancyhdr}
\pagestyle{fancy}
\fancyhf{}
\lhead{Algebra II - BASE}
\chead{Exponential Functions}
\rhead{AO Test, Page \thepage}
\usepackage{lastpage}
\usepackage{xcolor}
\usepackage{enumitem}
\usepackage{pifont}
\usepackage{graphicx}
\graphicspath{{../img}}
\usepackage{pgfplots}
\pgfplotsset{compat=1.18}
\usepackage{tabularx}
\usepackage{tikz}
\usetikzlibrary{patterns}
\usepackage{luacode}

\newcommand{\R}{\mathbb R}
\newcommand{\e}{{\rm e}}
\newcommand{\pobr}[1]{\left\langle#1\right\rangle}
\newcommand{\norm}[1]{\lVert #1 \rVert}
\newcommand{\abs}[1]{\lvert #1 \rvert}

\DeclareMathOperator{\xd}{d\!}
\DeclareMathOperator{\proj}{proj}

\title{}
\date{}

\begin{document}
\noindent
{\large
First Name \rule{6em}{.1pt}\hspace{\stretch{1}}Last Name \rule{6em}{.1pt}\hspace{\stretch{1}} Date \rule{1.5em}{.1pt} -- \rule{1.5em}{.1pt} -- \rule{1.5em}{.1pt}\hspace{\stretch{1}} Period \rule{2em}{.1pt}\hspace{\stretch{1}} Score \rule{2em}{.1pt}
}
\vspace{1em}

\begingroup
\renewcommand{\arraystretch}{1.5}
\begin{center}
\tiny
{
\begin{tabularx}{\textwidth}{|X|X|X|X|X|X|}
\hline
\bf MODEL & \centerline{Integrating} & \centerline{Applying} & \centerline{Practicing} & \centerline{Acquiring} & \centerline{Awaiting Evidence} \\
\hline
I can use math to model and solve real-world problems.&
Correctly identifies
important
quantities and
illustrates their
relationships using
diagrams, tables,
graphs, or
formulas.
Appropriate work is
shown with no
errors. The answer
includes units and
rounding as
appropriate to the
problem.
Explains how the
answer makes
sense in the
context of the
problem.
&Correctly identifies
important
quantities and
illustrates their
relationships using
diagrams, tables,
graphs, or
formulas.
Appropriate work is
shown with no
errors. The answer
includes units and
rounding as
appropriate to the
problem.
&Correctly identifies
important
quantities and
illustrates their
relationships using
diagrams, tables,
graphs, or
formulas.
Appropriate work is
shown with 1
COMPUTATIONAL
or ROUNDING
error.
&Correctly identifies
important
quantities and
attempts to
illustrate their
relationships using
diagrams, tables,
graphs, or formulas
Appropriate work is
shown with 1
CONCEPTUAL
error.
&Correctly identifies
important
quantities and
attempts to
illustrate their
relationships using
diagrams, tables,
graphs, or formulas
Appropriate work is
shown with more
than 1 conceptual
error.\\
\hline
\bf Criteria&\multicolumn{5}{l|}{\parbox[c][4em]{.8\textwidth}{}}\\
\hline
\end{tabularx}
}
\end{center}
\endgroup

\begin{luacode*}
math.randomseed(os.time())
car_price = 20000 + 500*math.random(0, 40)
depreciate_rate = math.random(80,90)/100
car_value = string.format("$f(t) = %d(%.2f)^t$", car_price, depreciate_rate)
car_year = math.random(3,8)
after_price = string.format("%.0f", car_price*.5)
home_price=600000 + 5000 * math.random(0, 60)
home_year=math.random(30,70)
appreciate_rate = math.random(2,5)
function home_value (year)
	return home_price*(1 + appreciate_rate/100)^year
end
home_value_between = string.format("%.0f", (home_value(home_year) + home_value(home_year+1))/2 )

aList = {}
bList = {}
exprList = {}
for i = 1, 4 do
	aList[i] = math.random(100, 2000)
	bList[i] =  math.random(1,99)
	exprList[i] = string.format("The %d\\%% of %d", bList[i], aList[i])
end

for i = 5, 6 do
	aList[i] = math.random(1000,10000)
	bList[i] = math.random(1,10)
end
\end{luacode*}
\vspace{1em}

\begin{enumerate}[leftmargin=*]
\item
\textit{\textbf{Instruction}: To support different learning needs, this assignment offers two problem choices, \textbf{Rebuild and Recap}.\\
\textbf{Solve only one}---choose the problem that you feel best demonstrates your understanding.\\
Full credit can be earned from either problem.}

\textbf{\large Rebuild:}
\begin{enumerate}
\item

Find the following quantities:

\parbox{.5\textwidth}{\begin{itemize}
\item
\directlua{tex.print(exprList[1])}
\vspace{8em}
\item
\directlua{tex.print(exprList[2])}
\end{itemize}}
\parbox{.5\textwidth}{\begin{itemize}
\item
\directlua{tex.print(exprList[3])}
\vspace{8em}
\item
\directlua{tex.print(exprList[4])}
\end{itemize}}
\vspace{\stretch{1}}
\item
Your investment of \directlua{tex.print(aList[6])} dollars has a return rate of \directlua{tex.print(bList[6])}\% per year. Complete the following table:
\vspace{1em}

\centering
\renewcommand{\arraystretch}{3}
\begin{tabular}{|c|c|}
\hline
 \bf Year&\bf Principal\\
 \hline
0 & \$\directlua{tex.print(aList[6])} \\
 \hline
 1 & \parbox{8em}{\hspace{\fill}}\\
 \hline
 2 &\\
 \hline
 3 &\\
  \hline
 4 &\\
\hline
\end{tabular}
\end{enumerate}
\clearpage 
\textbf{\large Recap:}

The graph shows a rabbit population that has been growing exponentially.

\begin{tikzpicture}
\begin{axis}[
axis lines=middle,
ylabel={$r$},
xlabel={$t$},
grid=both,
ymin=0,
ytick distance = 15,
xtick distance = 1,
width = .8\textwidth
]
\addplot[thick, domain=0:4]{60*(1.5^x)};
\end{axis}
\end{tikzpicture}
\begin{enumerate}
\item
What was the population when it was first measured?
\vspace{\stretch{1}}
\item By what factor did the population grow in the first year?
\vspace{\stretch{1}}
\item Write an equation relating the rabbit population, $r$, and the number of years since it was first measured, $t$.
\vspace{\stretch{1}}
\item What was the population after \textbf{18 months} since it was first measured?
\vspace{\stretch{1}}
\end{enumerate}
\clearpage

\noindent\leavevmode\leaders\hbox to 1em{\hss--\hss}\hfill\kern0pt \textit{PROBLEMS BELOW THIS LINE ARE MANDATORY FOR ALL} \leavevmode\leaders\hbox to 1em{\hss--\hss}\hfill\kern0pt
\item
The equation \directlua{tex.print(car_value)} represents the value of a car, in dollars, $t$ years after it was purchased.
\begin{enumerate}
\item
What was the purchasing price of the car?
\vspace{\stretch{1}}
\item 
What is the rate at which the car depreciates every year?
\vspace{\stretch{1}}
\item
How much is the car worth after \directlua{tex.print(car_year)} years?
\vspace{\stretch{1}}
\item 
Will the value of the car be more than \$\directlua{tex.print(after_price)} ten years after it was purchased? Show your calculation.
\vspace{\stretch{1}}
\end{enumerate}

\item A home is purchased for \$\directlua{tex.print(home_price)}. Since then, its value has increased \directlua{tex.print(appreciate_rate)}\% per year.
\begin{enumerate}
\item
Write an equation, in function notation, to represent the value of the home as a function of time in years since it was purchased, $t$.
\vspace{\stretch{1}}
\item Between which two years  after it was purchased does the home value reach \$\directlua{tex.print(home_value_between)}? Show your calculation.
\vspace{\stretch{1}}
\end{enumerate}

\end{enumerate}

\end{document}