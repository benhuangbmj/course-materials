\documentclass[twoside, 10pt]{article}

\usepackage{geometry}
\geometry{outer=3em, inner=2.2cm, top=6em, bottom=4em, headheight=\paperheight}
\usepackage[export]{adjustbox}
\usepackage{array}
\usepackage{amsmath}
\usepackage{amsfonts}
\usepackage{fancyhdr}
\pagestyle{fancy}
\fancyhf{}
\lhead{Algebra II - BASE}
\chead{Compositions of Functions}
\rhead{Practice, Page \thepage}
\usepackage{lastpage}
\usepackage{xcolor}
\usepackage{enumitem}
\usepackage{pifont}
\usepackage{graphicx}
\graphicspath{{../img}}
\usepackage{pgfplots}
\pgfplotsset{compat=1.18}
\usepackage{tabularx}
\usepackage{tikz}
\usetikzlibrary{patterns}

\newcommand{\R}{\mathbb R}
\newcommand{\e}{{\rm e}}
\newcommand{\pobr}[1]{\left\langle#1\right\rangle}
\newcommand{\norm}[1]{\lVert #1 \rVert}
\newcommand{\abs}[1]{\lvert #1 \rvert}

\DeclareMathOperator{\xd}{d\!}
\DeclareMathOperator{\proj}{proj}

\title{}
\date{}

\begin{document}
\noindent
{\large
First Name \rule{6em}{.1pt}\hspace{\stretch{1}}Last Name \rule{6em}{.1pt}\hspace{\stretch{1}} Date \rule{1.5em}{.1pt} -- \rule{1.5em}{.1pt} -- \rule{1.5em}{.1pt}\hspace{\stretch{1}} Period \rule{2em}{.1pt}\hspace{\stretch{1}} Score \rule{2em}{.1pt}
}
\vspace{1em}

\noindent
{\bf Learning Objectives.}
\begin{itemize}
\item
Implement the composition of two functions.
\item 
Implement the decomposition of a function into two simpler functions.
\end{itemize}
\noindent
{\bf Do Now.} {\it Work on the following problems when the teacher is taking attendance and returning works.}

Find the (natural) domain and range of the following functions.
\begin{enumerate}
\item
$\displaystyle f(x) = -\sqrt{x}$
\vspace{\stretch{1}}
\item
$\displaystyle g(x) =\sqrt{x^2}$
\vspace{\stretch{1}}
\end{enumerate}
\noindent
{\bf Practice.}
{\it In this practice, we review how to ``plug in a number'', namely, evaluate the function at a number.}

In the following problems, find the value of the function at the given point. If the value is undefined, put {\bf DNE} (Does Not Exist).
\begin{enumerate}
\item
$f(x) = x^3 + 4x^2 - 10$, find $f(2)$.
\vspace{\stretch{1}}
\item 
$\displaystyle g(x) = \frac{2}{x+6}$, find $g(10)$ and $g(-6)$.
\vspace{\stretch{1}}
\item 
$h(x) = 2x^2 - 3\sqrt{x}$, find $h(9)$ and $h(-9)$.
\vspace{\stretch{1}}
\end{enumerate}
\clearpage

\noindent
{\bf Discussion.}
{\it Composition is just about putting the output of a function/process to the input of another. It's used a lot to organize stuffs in not only mathematics, but also software engineering and almost everything. In other words, it's about building a complex structure from simpler ones. In contrast, decomposition is about break complex things into simpler ones.}

Discuss how the following process can be constructed by composing two sub-processes:
You are building a video game, and currently you are working on the AI of the enemies. The goal is simple: the enemy object wants to march toward the player once the player enters its sight. Ignoring all the implementation details of code, write this target process as a composition of two sub-processes. 

\noindent
{\bf Examples.}

Let $f(x) = x^2 +1$, $g(x) = x - 2$.
\begin{enumerate}
\item
Find $f(g(5))$.
\vspace{\stretch{1}}
\item 
Find $f(g(x))$. Simplify the answer.
\vspace{\stretch{1}}
\item 
Find $g(f(x))$. Simplify the answer.
\vspace{\stretch{1}}
\item
Find $(g(f(5))$.
\vspace{\stretch{1}}
\end{enumerate}
\noindent
{\bf Notations.}
\begin{itemize}
\item
``$(f\circ g)(x)$'' just means $f(g(x))$.
\end{itemize}
\noindent
{\bf Challenges.}
If $(f\circ g)(x) = \sqrt[3]{2x-5}$, what are $f(x)$ and $g(x)$?
\vspace{\stretch{1}}
\end{document}