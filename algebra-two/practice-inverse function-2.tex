\documentclass[twoside, 10pt]{article}

\usepackage{geometry}
\geometry{outer=3em, inner=2.2cm, top=6em, bottom=4em, headheight=\paperheight}
\usepackage[export]{adjustbox}
\usepackage{array}
\usepackage{amsmath}
\usepackage{amsfonts}
\usepackage{fancyhdr}
\pagestyle{fancy}
\fancyhf{}
\lhead{Algebra II - BASE}
\chead{Inverse Function, II}
\rhead{Practice, Page \thepage}
\usepackage{lastpage}
\usepackage{xcolor}
\usepackage{enumitem}
\usepackage{pifont}
\usepackage{graphicx}
\graphicspath{{../img}}
\usepackage{pgfplots}
\pgfplotsset{compat=1.18}
\usepackage{tabularx}
\usepackage{tikz}
\usetikzlibrary{patterns}

\newcommand{\R}{\mathbb R}
\newcommand{\e}{{\rm e}}
\newcommand{\pobr}[1]{\left\langle#1\right\rangle}
\newcommand{\norm}[1]{\lVert #1 \rVert}
\newcommand{\abs}[1]{\lvert #1 \rvert}

\DeclareMathOperator{\xd}{d\!}
\DeclareMathOperator{\proj}{proj}

\title{}
\date{}

\begin{document}
\noindent
{\large
First Name \rule{6em}{.1pt}\hspace{\stretch{1}}Last Name \rule{6em}{.1pt}\hspace{\stretch{1}} Date \rule{1.5em}{.1pt} -- \rule{1.5em}{.1pt} -- \rule{1.5em}{.1pt}\hspace{\stretch{1}} Period \rule{2em}{.1pt}\hspace{\stretch{1}} Score \rule{2em}{.1pt}
}
\vspace{1em}

\noindent
{\bf Skill Builder.} {\it Work on the Skill Builder problems on the screen while the teacher is taking attendance and returning work.}\\[.25em]

\noindent
\rule{\textwidth}{.1pt}\\[1em]
\rule{\textwidth}{.1pt}\\[1em]
\rule{\textwidth}{.1pt}\\[1em]
\rule{\textwidth}{.1pt}\\[1em]
\rule{\textwidth}{.1pt}\\[1em]
\rule{\textwidth}{.1pt}\\[1em]

\noindent
{\bf Learning Objectives.}
\begin{itemize}
\item
Find the inverse function of a linear or a simple non-linear function.
\end{itemize}

\noindent
{\bf Review Practice.}
The following problems review the technique of solving linear and simple non-linear equations from Algebra I.
\begin{enumerate}
\item Solve $3x + 2 = 8$ for $x$.
\vspace{\stretch{1}}
\item Solve $x^2 -  3 = 6$ for $x$ in $[0, \infty)$. What if $x$ is in $(-\infty, 0]$?
\vspace{\stretch{1}}
\item Solve $2 = \sqrt{-x} + 1$ for $x$.
\vspace{\stretch{1}}

\end{enumerate}

\end{document}