\documentclass[11pt]{beamer}
\usetheme{PaloAlto}
\usecolortheme{seahorse}
\setbeamertemplate{navigation symbols}{}
\setbeamertemplate{caption}[numbered]
%general package
\usepackage[utf8]{inputenc}
\usepackage{array}
\usepackage[english]{babel}
\usepackage{geometry}
\usepackage{tcolorbox}
\usepackage[export]{adjustbox}
\usepackage{graphicx}
\usepackage{xcolor}
\graphicspath{{../img}}
\hypersetup{
    colorlinks=true, 
    linkcolor=red,    % Internal links
    urlcolor=blue     % External links
}
\usepackage{wrapfig}
\usepackage{luacode}
\usepackage{comment}
\usepackage{tabularx}

%math package
\usepackage{amsmath}
\usepackage{amsfonts}
%\usepackage{amssymb}
%\usepackage{amsthm}
%\usepackage{slashed}
%\usepackage{tikz-cd}
\usepackage{pgfplots}
\pgfplotsset{compat=1.18}
\usepackage{tikz}
\usetikzlibrary{patterns}

%font package
%\usepackage{mathrsfs}
%\usepackage{bm}

%misc. package
\usepackage{enumitem}
\usepackage{animate}

\DeclareMathOperator{\xd}{\,d\!}
\DeclareMathOperator{\curl}{curl}
\DeclareMathOperator{\dive}{div}
\newcommand{\e}{{\rm e}}
\newcommand{\norm}[1]{\lVert#1\rVert}
\newcommand{\R}{\mathbb R}
\newcommand{\vF}{\mathbf F}
\newcommand{\vv}{\mathbf v}
\newcommand{\inpr}[1]{\left\langle#1\right\rangle}

\author[B.H.]{{\Large Unit Fraction Exponents}\\\vspace{.5em} Author: Benjamin Huang}
\date{}
\title[IM-Unit4-Lesson3]{Algebra II}

%\institute[]{\vskip -2em\includegraphics[width = 0.65\textwidth]{}}
%\logo{\includegraphics[width=.12\textwidth]{}}

\begin{document}
\begin{luacode*}
function generateNonzero(min, max)
  local output
  repeat
    output = math.random(min, max)
  until output ~= 0
  return output
end
\end{luacode*}
\frame{\titlepage}

\begin{frame}{Warm-up}
\Large
Have trouble getting started? Read the examples below:
\begin{enumerate}[label=(\arabic*)]
\item
$x^2 = 36\quad\Longrightarrow\quad x = \pm \sqrt{36} = \pm 6$
\vspace{2em}
\item 
$y^3 = 27\quad\Longrightarrow\quad y = \sqrt[3]{27} = 3$
\end{enumerate}
\end{frame}

\begin{frame}\frametitle{Learning Targets}
\Large
\begin{enumerate}[label = \arabic*.]
\item
I can write square and cube roots as exponents.
\end{enumerate}
\end{frame}

\begin{frame}
\frametitle{Review: Properties of Exponents}
\vskip -4em

\begin{figure}[h]
\includegraphics[width=.3\textwidth, left]{read-aloud.png}
\end{figure}
\Large
Let $a,b\ne 0$.
\begin{itemize}[label=$\bullet$]
\item
$a^ma^n = a^{m+n}$
\item
$\displaystyle \frac{a^m}{a^n} = a^{m-n}$
\item
$(a^m)^n = a^{mn}$
\item 
$(ab)^m = a^mb^m$
\end{itemize}
\end{frame}

\begin{frame}
\frametitle{Review: Exponential Growth Model}

Suppose that a population of 500 rabbits doubles every year.\pause
\[
P(t) = 500\cdot 2^t,\quad t \text{ measured in years.}
\]\pause
Questions:
\begin{enumerate}[label = (\alph*)]
\item
How to find the population after 1 year?\\ \pause $P(1) = 500\cdot 2^1$\pause
\item How to find the population after half a year?\\ \pause $P(\frac{1}{2}) = 500\cdot 2^{1/2}$ \pause
\item How to find the population after 4 months?\\ \pause $P(\frac{1}{3}) = 500\cdot 2^{1/3}$
\end{enumerate}
\end{frame}

\begin{frame}
\frametitle{Wait a Second}
\begin{figure}[h]
\includegraphics[width=0.35\textwidth]{wondering.jpg}
\end{figure}
However, what on earth is $2^{1/2}$ and $2^{1/3}$? What about $2^{1/n}$ for any integer $n$?
\end{frame}
\begin{frame}{The Graph}
\centering
\begin{tikzpicture}
\begin{axis}[
axis lines=middle,
grid = both,
grid style = {dashed},
xtick distance = 1,
ytick distance = 5,
]
\addplot[thick, blue]{2^x} node [pos=0.9, left] {$y = 2^x$};
\end{axis}
\end{tikzpicture}\pause

But, why?
\end{frame}

\begin{frame}{The Graph}
\centering
\begin{tikzpicture}
\begin{axis}[
axis lines=middle,
grid = both,
grid style = {dashed},
xtick distance = 1,
ytick distance = 5,
width = .9\textwidth,
]
\addplot[thick, blue]{2^x} node [pos=0.9, left] {$y = 2^x$};
\draw[fill=black] (0,1) circle (2pt) node[above left] {\small $(0,1)$};
\draw[fill=black] (1,2) circle (2pt) node[above] {\small $(1,2)$};
\draw[fill=black] (2,4) circle (2pt) node[above] {\small $(2,4)$};
\draw[fill=black] (3,8) circle (2pt) node[above left] {\small $(3,8)$};
\draw[fill=black] (4,16) circle (2pt) node[above left] {\small $(4,16)$};
\end{axis}
\end{tikzpicture}

But, why?
\end{frame}

\begin{frame}{The Graph}
\centering
\begin{tikzpicture}
\begin{axis}[
axis lines=middle,
grid = both,
grid style = {dashed},
xtick distance = 1,
ytick distance = 5,
width = .9\textwidth,
xmax = 5, xmin=-5,
ymax=32, ymin = 0,
]
\draw[fill=black] (0,1) circle (2pt) node[above left] {\small $(0,1)$};
\draw[fill=black] (1,2) circle (2pt) node[above] {\small $(1,2)$};
\draw[fill=black] (2,4) circle (2pt) node[above] {\small $(2,4)$};
\draw[fill=black] (3,8) circle (2pt) node[above left] {\small $(3,8)$};
\draw[fill=black] (4,16) circle (2pt) node[above left] {\small $(4,16)$};
\draw (0,1) -- (1,2);
\draw (1,2) -- (2,4);
\draw (2,4) -- (3,8);
\draw (3,8) -- (4,16);
\draw (4,16) -- (5,32);
\draw (0,1) -- (-1,1/2);
\draw(-1,1/2) -- (-2, 1/4);
\draw(-2,1/4) -- (-3, 1/8);
\draw(-3,1/8) -- (-4, 1/16);
\draw(-4,1/16) -- (-5, 1/32);
\end{axis}
\end{tikzpicture}

Why not?
\end{frame}

\end{document}