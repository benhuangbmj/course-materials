\documentclass[twoside, 10pt]{article}

\usepackage{geometry}
\geometry{outer=3em, inner=2.2cm, top=6em, bottom=4em, headheight=\paperheight}
\usepackage[export]{adjustbox}
\usepackage{array}
\usepackage{amsmath}
\usepackage{amsfonts}
\usepackage{fancyhdr}
\pagestyle{fancy}
\fancyhf{}
\lhead{Algebra II - BASE}
\chead{Exponents and Roots}
\rhead{AO Group, Page \thepage}
\usepackage{lastpage}
\usepackage{xcolor}
\usepackage{enumitem}
\usepackage{pifont}
\usepackage{graphicx}
\graphicspath{{../img}}
\usepackage{pgfplots}
\pgfplotsset{compat=1.18}
\usepackage{tabularx}
\usepackage{tikz}
\usetikzlibrary{patterns}
\usepackage{luacode}

\newcommand{\R}{\mathbb R}
\newcommand{\e}{{\rm e}}
\newcommand{\pobr}[1]{\left\langle#1\right\rangle}
\newcommand{\norm}[1]{\lVert #1 \rVert}
\newcommand{\abs}[1]{\lvert #1 \rvert}

\DeclareMathOperator{\xd}{d\!}
\DeclareMathOperator{\proj}{proj}

\title{}
\date{}
\input{"../learning-targets.local.tex"}

\begin{luacode*}
math.randomseed(os.time())
a11= math.random(-10,10)
a13 = math.random(-10,10)
a12 = math.random(2,30)
expr_a1 = string.format("\\(%d^{%d}\\cdot %d^{%d} = %d^a\\)", a12,a11, a12, a13, a12)
b11 = math.random(-10,10)
b12 = math.random(-10,10)
b13 = math.random(2,30)
expr_b1 = string.format("\\(\\displaystyle\\frac{%d^{%d}}{%d^b} = %d^{%d}\\)", b13, b11, b13, b13, b12)
c11 = math.random(2,10)
expr_c1 = string.format("$\\displaystyle %d^c = \\frac{1}{%d}$", c11, c11^2)
d1 = math.random(-5,5)
d2 = math.random(-5,5)
d3 = math.random(2,30)
expr_d1 = string.format("$\\left(%d^{%d}\\right)^d = %d^{%d}$", d3, d1, d3, d1*d2)
a21 = math.random(2, 9)
a22 = string.format("%d",a21^2)
expr_a21 = string.format("$\\displaystyle %d^{\\frac{1}{2}} \\cdot %d^{\\frac{1}{2}}$", a21^2, a21^2)

a30 = 3
a31 = a30^6
a32 = string.format("%d", a31)

a41 = math.random(2,5)
a42 = math.random(2,4)
if a42 == 3 then
	a42 = 2
end
expr_a4 = string.format("$%d^\\frac{%d}{3}$", a41^3, a42)

a51 = math.random(2,200)
a52 = math.random(2,200)
a53 = math.random(2,200)

a61 = math.random(2,9)
a62 = math.random(a61^2 + 1, (a61+1)^2 - 1)
expr_a6 = string.format("$\\sqrt{%d}$", a62)

a71 = math.random(2,9)
a72 = math.random(a71 + 1, a71 + 9)
a73 = math.random(1,10)
expr_a7 = string.format("$\\sqrt{x + %d} + %d = %d$", a73, a72, a71)

a81 = math.random(2,9)
a82 = math.random(2,20)
a83 = math.random(1,9)
expr_a8 = string.format("$\\frac{1}{%d}\\sqrt{%d + x} = %d$", a81, a82, a83)

a91 = math.random(2,5)
a92 = (2*math.random(2,10) + 1)/2
\end{luacode*}
\begin{document}
\noindent
{\large
First Name \rule{6em}{.1pt} \hspace{\stretch{1}}Last Name \rule{6em}{.1pt} \hspace{\stretch{1}} Date \rule{1.5em}{.1pt} -- \rule{1.5em}{.1pt} -- \rule{1.5em}{.1pt}\hspace{\stretch{1}} Period \rule{2em}{.1pt}\hspace{\stretch{1}} Score \rule{2em}{.1pt}
}
\vspace{1em}

\learningtargetbeprecise{For each group, a random paper will be collected and graded. The grade will then be applied to all present group members.}

\begin{enumerate}[leftmargin=*]
\item Complete the table. Use powers of \directlua{tex.print(a32)} in the top row and radicals or rational numbers in the bottom row.\\ (\textbf{Hint}: $\directlua{tex.print(a32)} = \directlua{tex.print(a30)}^6$. )
\begin{center}
\renewcommand{\arraystretch}{3}
\begin{tabularx}{0.65\textwidth}{|>{\centering\arraybackslash}X|>{\centering\arraybackslash}X|>{\centering\arraybackslash}X|>{\centering\arraybackslash}X|>{\centering\arraybackslash}X|>{\centering\arraybackslash}X|}
\hline
$\directlua{tex.print(a32)}^1$&$\directlua{tex.print(a32)}^\frac{1}{2}$& & $\directlua{tex.print(a32)}^0$& & $\directlua{tex.print(a32)}^{-1}$\\
\hline
$\directlua{tex.print(a32)}$& &\directlua{tex.print(a31^(1/3))}& & $\displaystyle\frac{1}{\directlua{tex.print(a31^(1/2))}}$ & \\
\hline 
\end{tabularx}
\end{center}
\item Complete the following problems without a calculator.
\begin{enumerate}
\item 
Evaluate $(\sqrt{\directlua{tex.print(a51)}})^2$.
\vspace{\stretch{1}}
\item
Evaluate $\left(\sqrt[3]{\directlua{tex.print(a52)}}\right)\left(\sqrt[3]{\directlua{tex.print(a52)}}\right)\left(\sqrt[3]{\directlua{tex.print(a52)}}\right)$.
\vspace{\stretch{1}}
\item Evaluate $\left(\sqrt[4]{\directlua{tex.print(a53)}}\right)\left(\sqrt[4]{\directlua{tex.print(a53)}}\right)\left(\sqrt[4]{\directlua{tex.print(a53)}}\right)\left(\sqrt[4]{\directlua{tex.print(a53)}}\right)$.
\vspace{\stretch{1}}
\end{enumerate}
\clearpage

\item Find solution(s) to each equation , or explain why there is no solution.
\begin{enumerate}
\item
\directlua{tex.print(expr_a7)}
\vspace{\stretch{1}}
\item
\directlua{tex.print(expr_a8)}
\vspace{\stretch{1}}
\end{enumerate}
\item Use the meaning of cube roots to find exact solutions to the following equations.
\begin{enumerate}
\item
\(\sqrt[3]{x} = \directlua{tex.print(a91)}\)
\vspace{\stretch{1}}
\item
\(\sqrt[3]{x} + \directlua{tex.print(a92)} = 0\)
\vspace{\stretch{1}}
\end{enumerate}
\end{enumerate}

\end{document}